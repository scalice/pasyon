\begin{abstract}
\noindent This critical re-examination of Reynaldo Ileto's 1979 work \textup{Pasyon and Revolution} argues that Ileto's attempt to reconstruct the categories of perception of \enquote*{the masses} using the \textup{pasyon} as source material was deeply flawed. Ileto treated the \textup{pasyon} as a literary text, ignoring the significance of its performance and treating it in an ahistorical manner. An attentiveness to performance reveals that the \textup{pasyon} was a cross-class and linguistically specific phenomenon. This insight dramatically attenuates the argumentative force of Ileto's claim to provide insight into the consciousness of the masses and their participation in revolution. Paying heed to the historical specificity of performance allows us to use other sources, such as the Bernardo Carpio legend and references to Tapusi to explore working class and peasant perceptions of revolution while avoiding the errors of Ileto's earlier attempt.
\end{abstract}

 
