\section{History}
\subsection{The explanatory primacy of objective historical circumstances}

To locate the categories of perception which informed the consciousness of the urban working class, agricultural day laborers, and the peasantry, we must look for sources other than pasyon and awit, and we must read in a manner altogether different from the manner of \textit{Pasyon and Revolution}. 

If we succeed in locating and interpreting these sources we may be able to begin to understand how the masses would have interpreted their role in the Philippine revolution. We will not, however, learn from this reconstructed consciousness why the masses revolted to begin with. To address this question we must address the historical circumstances that shaped working class and peasant consciousness and that made revolution for them an objective necessity.

Social banditry, the Carpio legend, and the Philippine revolution all emerged in the late nineteenth century out of the dramatic transformations which the working classes underwent. A rapid rise in population occurred beginning in the late eighteenth century and lasting until 1870.  Available resources were under tremendous strain. From 1870 until the century's end, \enquote{the archipelago suffered an unprecedented siege of crisis mortality.}\footcite[4]{Doeppers1998}

Commodity production took hold of every aspect of Philippine life; each region was transformed by capitalist economic relations and export production, most notably sugar, abaca, and tobacco crops.\footcites{Fast1979}{Owen1984}{DeJesus1980} Populations became intensely mobile in response to changing economic circumstances and shifting demands for labor power. This mobility was facilitated by steamboat and railroad transit.\footcite{Corpuz1999} Mass mobility in turn facilitated the spread of communicable disease. Smallpox, beriberi, malaria, and cholera devastated the human population. Rinderpest wiped out an entire generation of draft animals.\footcite{DeBevoise1995}

Concepts of space and time, the commonsense ways in which people measure their lives, inevitably shrank in response to this new, intensely mobile, and unpredictable world. These changes also rang the death knell for the central role of pasyon in everyday life; its expiring gasp may have been a long one, but gradually the incompatibility of the interminably slow performance of pasyon with the intensity of the demands of commodity production has reduced the pasyon to a cultural residue.

The latter half of the nineteenth century reveals enough evidence of \enquote{ongoing and episodic migration to demolish whatever may be left of the myth of the timeless Asian peasant rooted firmly to his ancestral lands.}\footcite[10]{Doeppers1998} Factory production transformed Manila. Huge masses of people, predominantly women, were employed as factory workers.

It was this mobility of the working population which gave mobility to the Carpio legend. It traveled down the San Mateo River with the workers ferrying timber from the mountains, into the Tabacera factories of Tondo with the women who migrated from San Mateo, out the Pasig River with market vendors on their return upstream, and throughout Laguna, Cavite and Bulacan.

This mobility not only facilitated the spread of the Carpio legend it also increased the banditry to which the legend pointed. Hobsbawm wrote, \enquote{What makes peasants the victims of authority and coercion is not so much their economic vulnerability -- they are indeed as often as not virtually self-sufficient -- as their immobility. Their roots are in the land and the homestead, and there they must stay like trees \ldots\ If we want to understand the social composition of banditry, we must therefore look primarily at the mobile margin of peasant society.}\footcite[34-5]{Hobsbawm2000} 

The social and economic transformations of the nineteenth century simultaneously destroyed peasant self-sufficiency -- leaving the rural population intensely vulnerable to changes in the world market -- and forced upon them a new, intensely mobile life, one without roots in any particular piece of land or homestead. It is no surprise that banditry became rampant during this time. What is important is that many sections of the populace moved beyond this traditional response to an antiquated and collapsing social system and looked instead to the revolutionary anti-colonial politics of the Katipunan.

Isabelo de los Reyes wrote in 1899, 

\begin{quote}
During the past decade the country has been suffering a business recession that has deteriorated the last years. Indigo production is completely paralyzed, and hemp and sugar prices have fallen so much that they can scarcely cover costs. A canker has attacked the coffee plantations and coffee has disappeared from the market. Only rice, which is precisely the article of prime necessity, being the staple food of the Filipinos, has risen in price; and, because of the unfavorable exchange, imported goods.

To this must be added the fact that in June and July of 1896 thick swarms of locusts completely ruined the rice fields, and farmers faced a future that was bleak indeed. They already groaned under the hard yoke of the friar \textit{hacenderos}, who far from remitting even a part of ground rent in consideration of the low prices, the locust plague and the drought, steadily increased it; and so the peasants, driven to desperation, swelled the ranks of the revolution.\footnote{Isabelo de los Reyes, as quoted in \cite[205]{DelaCosta1965}.}
\end{quote}

Pasyon and Revolution examines the perceptions of the \enquote*{masses} and finds superstitions, amulets, and \enquote{a society where King Bernardo Carpio was no less real than the Spanish governor-general.} Ileto examined the masses' worldview, demonstrating that it was internally coherent, and possessed its own rationality, or counter-rationality. He took the elite conceptions about the \enquote*{pobres y ignorantes} and stood them upon their heads, transvaluing them. Ileto did not, however, undermine these conceptions.

Attentiveness to actual class relations and to the oral nature of peasant literature reveals a very different picture. Here we can see a deep seated historicity to lower class discourses, conducted in a register designed to occlude these discourses from elite perception and interference.

\subsection{Conclusion: Bonifacio and Aguinaldo}

This can be concretely demonstrated by the examination of a simple question: why did the 'masses' follow Bonifacio and not Aguinaldo? Why did Aguinaldo never gain the popular support that Bonifacio had?

Aguinaldo was a religious man who held Bonifacio's secular worldview in contempt. He spoke a pasyon inflected language, with far greater fluency than Bonifacio. Bonifacio, to the best of our knowledge, never had any anting-anting; Aguinaldo had several. \textit{Pasyon and Revolution} cites an article from an August 1897 edition of the \textit{New York Herald} which stated

\begin{quote}
Among other followers he \textins{Aguinaldo} had two youths appropriately dressed as pages who accompany him everywhere and who seemed to be considered as persons of no little importance by the others. One of the youths in particular has attracted attention which is explained by others of his followers in this way. This interesting youth possesses the supernatural qualities of anting-anting. (26)
\end{quote}

Finally, and this cuts to the heart of \textit{Pasyon and Revolution}'s categories of class, Aguinaldo, the ilustrado, could not read or write in Spanish. In fact, he took little interest in his education at all. He confessed late in life that he had never read Rizal's \textit{Noli me Tangere} or \textit{El Filibusterismo}.\footcite[144]{Ocampo2001} This was true not only of Aguinaldo but of many ilustrados. For \enquote*{ilustrado} to be a class category it must cease to mean enlightened or educated and must simply refer to a mestizo owner of the means of production. 

Bonifacio, the self taught urban worker, could read Spanish. He had read Rizal, Victor Hugo, Eugene Sue, \textit{Lives of the Presidents of the United States}, and books on the French Revolution. He translated Rizal's final poem, \textit{Mi Ultimo Adios}, into Tagalog. He was an educated man.

One of the few definitions of the lower classes that Ileto gives us is on the fourth page of \textit{Pasyon and Revolution}. There he glosses the \enquote*{masses} as \enquote{the largely rural and uneducated Filipinos who constituted the revolution's mass base.} (4) If we ignore the anachronistic use of the word \enquote*{Filipino} we are left with a definition that would clearly make Aguinaldo one of the masses and Bonifacio not.

Why then did the masses not passionately follow Aguinaldo? Why did they so identify with Bonifacio? Perhaps one more example from \textit{Pasyon and Revolution} will help us to answer this question. 

On the cover of \textit{Pasyon and Revolution} is a drawing, \enquote{an artist’s composite based on a statue housed in the Santa Clara church (of the Philippine Independent Church) in Sampaloc, Manila, and on Aurelio Tolentino's story of a Katipunero's dream of the Virgin.} This story of the dream of the Virgen sa Balintawak forms part of the argument of \textit{Pasyon and Revolution}. An article was published in \textit{La Vanguardia} sometime after the death of Aurelio Tolentino, the Katipunero and playwright, in 1915. It told of a dream in which the Virgin Mary appeared to Bonifacio in native dress and warned him of betrayal. According to the story, Bonifacio acted on the warning and thus avoided arrest. Ileto admitted \enquote{the story may be entirely apocryphal,} but, he continued, \enquote{[t]he point is, such a story was entirely credible to Tolentino's audience. Why was the Virgin in native costume; why was she leading a Katipunero by the hand? Was she the Mother Country herself? For the popular mind there was no clear distinction, no crisis of meaning as one image flowed into the other.} (106)

Who exactly were Tolentino's audience, these people of \enquote*{the popular mind?} \textit{La Vanguardia} was a Manila based Spanish language paper, which began publication in 1910 in the wake of the closure of \textit{El Renacimiento}. The \enquote*{masses} did not read this paper. It was not in \textit{La Vanguardia} that Ileto located this story, however. He found it printed on the back of the novenario of the Iglesia Filipina Independiente, \textit{Pagsisiyam sa Virgen sa Balintawak}.\footcite{AglipayYLabayan1925} The novena, a small book of prayers to be prayed over a period of nine days, is a remarkable one. While \textit{Pasyon and Revolution} discussed the one page story that formed part of its back matter, it did not touch upon the actual contents. They are worth examining.

Each day's prayer is preceded by a reading. The readings are an introduction to a rationalist approach to religion and to the world. The readings for the second day detail the evolution of life on earth over millions of years, deliberately and explicitly contradicting biblical creation myths. Day three states \enquote{salita salita lamang ang mga sinasabi ukol sa para\'iso, at sa mga angel, sa infierno, sa purgatorio, sa limbo at sa demonio.}\footcite[21]{AglipayYLabayan1925} \enquote{What is said about Paradise, angels, hell, purgatory, limbo and demons, is just empty words.} The novena examines scientific ideas of the end of the Universe and rebuts notions of the apocalypse. It refutes both Jesus' virgin birth and divinity, and states that he was \textit{kayumanggi}, the brown skin color of indios. \enquote{Nang gabi rin na siya'y mabilanggo, ay ipinayo niya sa kaniyang mga tao na gamitin hanggang kaliit-liitang c\'entimo upang bumili ng sandata, kaya't siya'y nahatulang mamatay, alinsunod sa batas ng mga romano dahil sa paglaban sa pamahalaan ng Roma.}\footcite[35]{AglipayYLabayan1925} \enquote{On the very night in which he was arrested, he told his followers to use their last centavo to buy arms, and he was condemned to death under Roman law for having rebelled against Roman sovereignty.} The Iglesia Filipina Indepediente, founded in 1902, as a nationalist, separatist church by Aglipay and Isabelo de los Reyes grew rapidly. Workers and peasants swelled its numbers to one and half million members. 

Try as one might to find it, there is no pasyon idiom to be located in the early documents of the Iglesia Filipina Indepediente; there are no myths of paradise, fall and redemption; no mysterious liwanag, damay, or lo\'ob to be found. The IFI was founded on militant politics and, under the leadership of Aglipay and de los Reyes, it openly advocated for immediate independence and for socialism. It did so in terms that, while Tagalog, would be unfamiliar to the reader of \textit{Pasyon and Revolution}. Rather than a static and atavistic idiom which structured the worldview of the \enquote*{masses} from 1840 to 1912 and far beyond, what we see here is a dynamic ideology of resistance which was consonant with the objective lived experience of the working class and of agricultural laborers. It was this consonance, this apt articulation of objective experience, which appealed to the \enquote*{masses} in the IFI.

Bonifacio did not speak a \enquote*{pasyon idiom.} He did not identify with a King hidden in a cave in whom the masses credulously believed. Bonifacio did not wear amulets to ward off bullets. He did, however, articulate the inchoate strivings of revolutionary sections of the peasantry and of the emergent working class better than almost anyone of his generation. This fact explains both his success as a leader and his death at the hands of the landowning class of ilustrados.
