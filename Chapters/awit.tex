\section{Pasyon and Awit}

\subsection{Pasyon and Revolution Revisited}

The Philippine revolution has been explained as having been inspired by the ideas acquired by the \textit{ilustrados}, the \textit{mestizo} elite, during their education abroad. These ideas led to the formation of freemasonry in the Philippines which, in turn, gave revolutionary inspiration to a lower-middle class clerk, Andres Bonifacio. Bonifacio founded the Katipunan, a separatist secret society, and led the opening salvos of the revolution. Control of the revolution eventually passed to Emilio Aguinaldo, who had Bonifacio tried and executed. This succession of power was seen by traditional scholarship as either regrettable but necessary, or as the usurpation of the reins of the revolution by the upper classes.\footnote{The name \textit{ilustrado} means \enquote*{the enlightened;} this is treated by most scholars as a class label for the educated \textit{mestizo} elite. I will interrogate the usefulness of the term \textit{ilustrado} in this paper. The standard works on the Philippine Revolution are \cite{Agoncillo1956}; \cite{Agoncillo1960}; \cite{Kalaw1969}; and \cite{Zaide1939}.}

\textit{Pasyon and Revolution} studied the ideas and events of the revolution differently, examining the history of Tagalog lower-class movements from 1840 to 1912. This was a time punctuated by both millenarian peasant uprisings and revolutions against Spain and the United States. Earlier scholars had treated these peasant uprisings as separate, local events, with no serious connection to the Katipunan or the Philippine Revolution. Ileto argued that by looking at these events as they would have been perceived by the masses\footnote{Ileto's use of class categories is deeply problematic, as I shall discuss in this paper. For purposes of simplicity, I shall continue to use his phrase, the \enquote*{masses,} when reconstructing and interacting with his argument. When outlining an alternative hypothesis I shall use a more specific vocabulary to understand the class structure of Philippine society.} we could see that the seemingly unconnected and irrational uprisings of the peasantry formed a coherent whole, seamlessly interwoven with the Philippine Revolution. To demonstrate this connection Ileto needed to recreate the ways in which the masses perceived the world. He wrote, \enquote{the physical involvement of the masses in the revolution was pretty clear, but how did they actually perceive, in terms of their own experience, the ideas of nationalism and revolution brought from the West by the ilustrados?} (4)

What was needed, according to Ileto, were \enquote{alternative, valid meanings} (7) for nationalism, independence, and revolution, meanings which would have been intelligible to the masses and corresponded to their understanding of the world. Without this understanding, peasant movements appear to be irrational and backwards.

To locate these alternative meanings, Ileto looked for new sources and read old sources in new ways. He reread the documents of the revolution with an eye to peasant and lower-class categories of perception, asking the reader to imagine how the masses would have understood them. Through a close study of \textit{awit}, Tagalog folksongs, and \textit{pasyon}, the sung version of the passion and death of Jesus Christ, \textit{Pasyon and Revolution} aimed to \enquote{arrive at the Tagalog masses' perceptions of events.} To do so, Ileto argued, \enquote{we have to utilize their documents in ways that extend beyond the search for \enquote{cold facts.}} (10)

The pasyon libretto which Ileto examined, \textit{Pasyon Pilapil} was first published in Tagalog in 1814. It is stylistically the roughest of the three available versions of the Pasyon; it was also the most popular. The pasyon was composed with the intent of inculcating submission and passivity into the colonized populace, and yet, Ileto argued, it contained passages which allowed them to identify their suffering with that of Christ. \enquote{Whether the pasyon encouraged subservience or defiance, resignation or hope, will always be open to argument. The fact is that its meanings are not fixed, but rather depended on social context. Thus a historical approach is necessary.} (18)

The pasyon gave the masses an idiom for articulating an understanding of the world; it did not provide them with an ideology or a coherent picture of society. This idiom comprised powerful units of meaning, located at the intersection of the masses experience of reality and their participation in the \textit{pabasa}, the sung performance of the pasyon. These units of meaning informed how the masses interacted and participated in society.

Ileto's study of the pasyon located these basic units of meaning: the pasyon conveyed \enquote{an image of universal history,} (14) structured as paradise, fall, redemption, and judgment. A section of the pasyon narrated Jesus' separation from his mother in response to a call \enquote*{from above.} This separation from family \enquote{probed the limit of prevailing social values and relationships} (14) and \enquote{paved the way} for \enquote{indios \ldots\ joining a rebel leader who was often a religious figure himself.} (15) Finally, Jesus called his followers from the \enquote{lowly, common people,} (16) and was persecuted by the wealthy and the powerful.

The narrative of peasant uprisings begins in a chapter entitled \enquote{Light and Brotherhood} which tells the story of the Cofrad\'ia de San Jose, a religious sodality founded by Apolinario de la Cruz, a charismatic peasant leader known as Hermano Pule. Stung by the Spanish religious orders’ rejection of his application for the recognition of his confraternity, de la Cruz responded by banning non-indio membership in the organization. Alarmed, the authorities moved to shut down the meetings of the Cofrad\'ia, sending soldiers to break up the organization. On the slopes of Mount San Cristobal, the members of the confraternity battled the soldiers for ten days. Three to five hundred members of the sodality were killed and another three to four hundred arrested.

Ileto was particularly interested in the perceptions of the Cofrad\'ia. To reconstruct the mentality of this group, he read through the hymns and prayers of the sodality and the letters which Apolinario de la Cruz addressed to its members. A constellation of Tagalog words formed out this examination which Ileto claimed were in keeping with pasyon idiom: \textit{liwanag}, radiant light which brings wisdom or insight; \textit{awa}, pity, which evokes a response of \textit{damay}, fellow feeling, which has the added significance of participating in another's work; \textit{layaw}, love, pampering, the satisfaction of necessities; and \textit{lo\'ob}, the interior of a thing, a person's will and emotions. \textit{Pasyon and Revolution} drew a distinction between historical time, the tangible events affecting the lives of the masses, and pasyon time, the deeper, invisible structure to history in which suffering, death, and redemption give everyday struggles a profound significance. Ileto claimed that the revolt of the Cofrad\'ia was an attempt to synchronize historical time with pasyon time; it was the irruption of the \enquote*{pasyon world} into the \enquote*{everyday world.}

The connection between the Cofrad\'ia de San Jose and later events was neatly summarized in \textit{Pasyon and Revolution}: \enquote{The events that culminated in the bloody revolt of 1841 was \textins{\textit{sic}} not simply a blind reaction to oppressive forces in colonial society; it was a conscious act of realizing certain possibilities of existence that the members were made conscious of through reflection upon certain mysteries and signs. Furthermore, since what we are talking about is part of \textit{the world view of a class of people with a more or less common religious experience}, the connection between the events of 1840-1841 and later upheavals in the Tagalog region can be posited.} (30, emphasis added)

In this paragraph there are two explicit assumptions underlying the argument for continuity between the Hermano Pule uprising and subsequent revolts, including that of the Katipunan. The first is that the uprising of 1841 emerged from \enquote*{the world view of a \textit{class} of people.} The second is that this class had a \enquote*{more or less common religious experience.} Ileto continued, \enquote{certain common features of these upheavals, or the way these events were perceived, indicate that connections do exist. These lie perhaps, not in a certain chain of events, but in the common features through time of a consciousness that constantly seeks to define the world in its own terms.} (31) The continuity in the history of revolts is thus the result of continuity in consciousness. Whose consciousness? The consciousness of a class. \textit{Pasyon and Revolution} argued that there was a continuity of class consciousness which was the basis of the continuity of the uprisings from 1840 to 1912. But the consciousness of what class? This is a question to which I shall return.

The next chapter, entitled \enquote{Tradition and Revolt: The Katipunan,} carries the narrative forward fifty years to the founding of the Katipunan, moving from 1841 to an event which occurred in 1897 in Tayabas. The remnants of Hermano Pule's Cofrad\'ia, now known as the Colorum Society, were led by Sebastian Caneo in a large procession entering the provincial capital. They came with the intention of throwing pieces of rope at the \textit{guard\'ia civil}, whom, they believed, would be magically tied up. The \textit{guard\'ia civil} opened fire on the procession, killing many. The rest fled.

This story, rather than the classic narrative of western ideas and ilustrado agitation for reform, forms the background of \textit{Pasyon and Revolution}’s account of the Katipunan uprising. Ileto stated, \enquote{the fact that a self-educated lower middle class clerk named Andres Bonifacio founded the Katipunan in 1892, is excessively attributed to the influence of ilustrados like Del Pilar and Rizal.} (79) What we should instead look for was \enquote{a way of reconstructing the masses' perceptions of the Katipunan and their role in it.} In order to do so, we needed to \enquote{cease for the moment to regard the Katipunan as a radically unique phenomenon or as the mere creation of individuals like Bonifacio and Jacinto.} (81)

Ileto examined the manifestos published by Bonifacio, Emilio Jacinto, and Pio Valenzuela in March, 1896 in the sole edition of the Katipunan circular \textit{Kalayaan}. A thousand copies circulated among readers in Manila, Bulacan, and Cavite. Between March and the discovery of the Katipunan by Spanish officials in August the membership grew from three hundred to twenty or thirty thousand.\footnote{PAR, 82, dates the publication of \textit{Kalayaan} to January 1896. This was the date on the masthead of the paper, but the paper was not completed until mid-March. For circulation and membership numbers, PAR cites Valenzuela's claim that 2,000 copies circulated and that 30,000 members joined. The more conservative estimates are given by E. de los Santos. For information on \textit{Kalayaan}, see \cite{Richardson2005}.}  Ileto found that the form and language of Bonifacio's main article in \textit{Kalayaan}, \enquote{Ang dapat mabatid ng mga Tagalog} (What the Tagalogs should know), were more important than its content for the purposes of his analysis. Bonifacio, he argued, communicated in his article by using \enquote{the pasyon form.} (83) 

Ileto also examined the initiation rituals of the Katipunan. These \enquote{appear to be Masonic. But if they were truly so,} Ileto asked rhetorically, \enquote{could unlettered peasants have embraced the Katipunan as truly their own?} (91) \textit{Pasyon and Revolution} demonstrated that both Bonifacio's article and the initiation rituals of the Katipunan were structured around the ideas of paradise, fall and redemption, or of pre-colonial prosperity, the advent of the Spaniards, and restoration through kalayaan. Ileto studied the etymology of this last word, which was of such central importance to the Katipunan that they made it the title of their paper, and found that it signified \enquote*{satisfaction of needs} and not simply autonomy.\footnote{It is interesting to note, however, that the word kalayaan was first used in a political context by the ilustrado Marcelo del Pilar in 1882 to translate the Spanish \enquote*{libertad.} In 1891, Rizal used kalayaan in translating \textit{The Declarations of the Rights of Man and the Citizen} to convey \textit{libert\'e} in Tagalog. \cite{Richardson2005}}

\textit{Pasyon and Revolution} moved from the study of initiation rituals to an overlooked episode in the founding of the Katipunan: the sojourn of Bonifacio and eight other leaders during Holy Week, 1895, to \enquote{Mount Tapusi} in preparation for the uprising. There, in the legendary cave of the folk hero Bernardo Carpio, they wrote on the wall, \enquote{Long live Philippine independence!} This journey, \textit{Pasyon and Revolution} stated, had \enquote{two levels of meaning. On one hand, it was purely military, a search for a haven. On the other, it was a gesture of identifying with the folk hero entombed in the mountain.} (102)
 
The chapter concludes with the power struggle between Bonifacio and Aguinaldo that emerged when Bonifacio traveled to Cavite, and culminated in his execution on May 10, 1897. Bonifacio demonstrated, in both his writings and his actions, \enquote{his familiarity with popular perceptions of change. Folk poetry and drama undoubtedly provided him with basic insights into the \enquote{folk mind.} Between him and Apolinario de la Cruz in fact exists a strong affinity.} These insights led Bonifacio to his \enquote{preoccupation with \enquote{sacred ideals} and moral transformation,} and it was this preoccupation which in turn led to his downfall. (109)

In its later chapters, \textit{Pasyon and Revolution} examined the use of the language of the Katipunan in the radicalism of the masses during the republican phase of the Philippine revolution, the Philippine American War, under Macario Sakay's revived Katipunan, and in the Santa Iglesia of Felipe Salvador. The masses continued to conceive of the revolution, and their role in it, in the pasyon form used in the early documents of the Katipunan under Bonifacio. \enquote{This phenomenon can be understood if we view Bonifacio's Katipunan as the embodiment of a revolutionary style, a sort of language which enabled the ordinary Indio to relate his personal experience with the \enquote{national.}} (113)

Ileto neatly summarized his argument, \enquote{the continuity in form between the Cofrad\'ia in 1841, the Katipunan revolt of 1896, the Santa Iglesia and other movements we have examined can be traced to the persistence of the pasyon in shaping the perceptions of particularly the poor and uneducated segments of the populace. Through the text and associated rituals, people were made aware of a \textit{pattern of universal history}. They also became aware of \textit{ideal forms of behavior} and social relationships, and a way to attain these through suffering, death, and rebirth.} (254, emphasis added)

The pasyon gave the masses \enquote{a pattern of universal history} -- that is, the pattern of paradise, fall, and redemption -- and \enquote{ideal forms of behavior} -- \textit{damay}, \textit{awa}, and so on. This idiom enabled the masses to understand the world, the revolution, and their participation in it.

\subsection{Problematic Class Categories} 

The historical continuity that \textit{Pasyon and Revolution} found in popular movements from 1840-1910 was a continuity of class consciousness. Before we can examine in detail the sources which Ileto used to reconstruct this class consciousness we must first ask what class or classes make up the \enquote*{masses,} the \enquote*{underside of Philippine history.}\footnote{The phrase is from \cite{Ileto1982}.}

The relationship between classes changed dramatically in the nineteenth century Philippines. The galleon trade between Manila and Acapulco ended in 1815. The Philippines' status as a colonial backwater, no more than a hub in trade with China, gradually ended as well. Pre-capitalist relations of production were overthrown by the introduction of foreign, largely British, capital, between the first to the second half of the nineteenth century. What class relations had prevailed in 1841 Tayabas during the Hermano Pule uprising would have borne pale semblance to those of the Katipunan's Tondo fifty years later. 

By the 1880s and 90s, Philippine society was awash in class contradictions. Small landholders, tenant farmers, share croppers, agricultural wage workers, an urban proletariat, clerks and professional wage workers -- all of these groups jostled uneasily with each other under indefinite rubrics in Ileto's account. \textit{Pasyon and Revolution} lumped these classes together as \enquote*{the masses;} the underside; those \enquote*{from below;} the poor; peasants; the \enquote*{common \textit{tao};} the \enquote*{illiterate \textit{tao};} they collectively share the \enquote*{popular mind;} the \enquote*{folk mind;} they are occasionally \textit{indios} who share \enquote*{the Filipino mind;} they are, quite often, simply the ilustrados' \enquote*{\textit{pobres y ignorantes}.} 

These categories are troubling. \textit{Pasyon and Revolution} introduced the phrase \textit{pobres y ignorantes} as \enquote{the common ilustrado term for the masses,} (18) yet never questioned the validity of the ilustrado characterization of the classes with which it was dealing. The \enquote*{masses} in \textit{Pasyon and Revolution} are a superstitious, illiterate lot. Ileto sought the categories of perception of these \textit{pobres y ignorantes}; he did not, however, question that they were and are backward. This is particularly evident in the introduction to \textit{Pasyon and Revolution}:

\begin{quote}
We modern Filipinos \ldots\ can either further accelerate the demise of \enquote{backward} ways of thinking (reflected in the Lapiang Malaya) in order to pave way for the new, or we can graft modern ideas onto traditional modes of thought. Whatever our strategy may be, it is necessary that we first understand how the traditional mind operates, particularly in relation to questions of change. This book aims to help bring about this understanding. (2)
\end{quote}

Despite the scare quotes around backward in the above quotation, the masses' modes of thought are clearly pre-modern in Ileto's conception; to belong to the masses is to possess a \enquote{traditional mind.} Without a clear sociological definition of the class or classes to which it referred, \textit{Pasyon and Revolution} began with the ilustrado notion of \enquote*{pobres y ignorantes} and then asked what consciousness this \enquote{group} possessed. 

Actual class relations were exceedingly volatile in the nineteenth century Philippines. New classes emerged, old classes disappeared. Subsistence agriculture gave way to cash cropping and commodity production. While class consciousness is notoriously viscous and lags behind objective circumstances, the transformations wrought by capitalism in the decades leading up to the Philippine revolution would have had profound effects on the consciousness of workers and peasants. The uninterrupted continuity in categories of perception that Ileto found stretching from Apolinario de la Cruz in the 1840s to Valentin de los Santos in 1967 thus warrants a healthy amount of suspicion.\footnote{Valentin de los Santos was a charismatic Bikolano religious leader who led a group of Southern Luzon peasants in a group called Lapiang Malaya in 1967. The Lapiang Malaya marched against the Marcos' government armed with bolos and anting-anting and were shot down by the constabulary on Taft avenue in May 1967. It was the month that Ileto departed the Philippines for graduate school at Cornell.}

\subsection{An Elite Textual Hermeneutic}

\textit{Pasyon and Revolution} studied two main sources to discover the consciousness of the masses: \textit{awit}, Tagalog folksongs; and pasyon, the sung version of the passion of Christ. Ileto argued that a close examination of this nineteenth century literature of the masses could reveal the ways they perceived the world. Ileto read these sources as texts in which the basic unit of meaning is the lexeme and allusions are intertextual. This is precisely how an ilustrado would have read pasyon or awit. It does not give us a sense of how peasants and workers would have understood them. In order to address lower class consciousness through an analysis of these works, we must read them in a different manner altogether; we must concern ourselves with their \textit{performance}.

Individual words are the fundamental units of meaning in \textit{Pasyon and Revolution}. They are the masses' categories of perception which we find in both pasyon and awit. Words like \textit{layaw}, \textit{damay}, \textit{awa}, \textit{lo\'ob}, and \textit{liwanag} seem profound to the non-native speaker and circulate untranslated throughout \textit{Pasyon and Revolution}. They acquire a reified sense of meaning far out of keeping with their actual workaday significance.  Thus we commonly read sentences like \enquote{Since \textit{damay} is a manifestation of a whole and controlled \textit{lo\'ob}, the Katipunan’s \textit{lo\'ob} radiates heat and flame, just as Christ and other individuals of exemplary \textit{lo\'ob} radiate \textit{liwanag}.} (136) The italicized words fly fast and thick and give the portentous feeling of significance. They attain a magical status: academic anting-anting which render \textit{Pasyon and Revolution} impervious to scholarly criticism.   

Ilustrados are not afforded the privilege of communicating in deeply meaningful untranslated words. When Aguinaldo appeals to \enquote*{banal na kalayaan,} in a proclamation addressed to workers not to go on strike during the war against the Americans, the phrase is translated as \enquote{sacred independence} and no original is included. Ileto dismissed this \enquote*{sacred liberty} as an \enquote{abstract notion.} (124)

This italicized, untranslated Tagalog clings, however, to the speeches of Bonifacio, Jacinto, and other working class leaders even when delivered in Spanish. The katipunero Aurelio Tolentino wrote \enquote{Viva la Independencia Filipina!} on the wall of the cave of Bernardo Carpio. Ileto rendered the text in English and then extrapolated the full Tagalog significance of the phrase. The passage is representative of the hermeneutical style of \textit{Pasyon and Revolution}, and is worth reproducing:

\begin{quote}
We can also understand why Bonifacio's hand trembled with fierce emotion as he wrote on the walls of the cave: \enquote{Long live Philippine independence!} This slogan must be interpreted in its entire form -- \textit{Panahon na! Mabuhay ang Kalayaan!} -- which was the battlecry of the Katipunan. Its common translation as \enquote{The time has come! Long live Liberty!} does not quite capture its meaning. \textit{Panahon na!} (It is time!) implies, not only that the revolution has begun, but that a totally new era (panahon) is about to succeed the old which has irreversibly winded down. And \textit{Mabuhay} should literally be translated literally as \enquote{May it live} or \enquote{May it come to life.} \enquote{Long live} or \enquote{cheers} fails \textins{\textit{sic}} to capture the meaning of the struggle as the experience of hardship in order to redeem or give life to a \enquote{dead} or \enquote{slumbering} condition called kalayaan. (103)
\end{quote} 

All of this was derived from four Spanish words written by Aurelio Tolentino.\footnote{And not, as Ileto claims, by the trembling hand of Bonifacio. \parencite[225]{Kalaw1965}.} Upon this slight foundation, \textit{Pasyon and Revolution} builds a comparison between the \enquote{\enquote{slumbering} condition called kalayaan} and Bernardo Carpio, whom the masses supposedly saw Bonifacio awakening, an interpretation addressed later in this paper.

If we do not employ an elite textual hermeneutic, but rather attend to the significance derived from the public performance of pasyon and awit, what insights do we gain?

\subsection{Pasyon as Performance}

The text of the passion of Jesus Christ was first translated into Tagalog by Gaspar Aquino de Belen in 1703.\footnote{On the performance of pasyon see \cite{Trimillos1992}; and \cite{Tiongson1975}.} This pasyon began with the Last Supper and continued through the death of Christ. In 1814, a second pasyon was written which was known as Pasyon Genesis or Pasyon Pilapil. The latter name was the result of the popular attribution of authorship to Father Mariano Pilapil, who submitted the document for imprimatur in 1884.\footcite[8]{Trimillos1992} Pasyon Pilapil begins with the creation of the world and concludes with the coronation of Mary in heaven. It was performed in two separate Lenten folk rituals: \textit{pabasa} and \textit{sinakulo}.

Pabasa is the public singing of the pasyon.\footnote{The pabasa is sung in a \textit{kapilya} (small chapel) or in private homes where shelters are constructed specifically for the purpose. The pasyon is not performed within the church because of the belief that it should not be sung where the host, the communion wafer, is present.} These performances were sponsored by prominent families and reinforced local hierarchy. The sponsoring family dictated the order of singers and could hire semi-professional pasyon performers. The performance of the pasyon was an extended event occupying the space of several days during Holy Week. Refreshments were provided in keeping with the Spanish colonial meal structure: \textit{desayuno}, \textit{caf\'e}, \textit{almuerzo}, \textit{merienda}, \textit{cena}, and \textit{caf\'e de noche}.

The audience came and went, talking loudly and eating during the performance. Sections of the pasyon varied in popularity. Audience interest tended to wane with the singing of the story of Cain and Abel, the lineages of Christ, or the \textit{aral}, homilies addressed directly to the audience. It is important to note that much of the vocabulary of Ileto's pasyon idiom is derived from the \textit{aral}, the least popular, most ignored sections of a performance. Christ's walking on water and his encounter with Mary on the \textit{Via Dolorosa}, with their magic and drama, were popular and well attended.\footnote{There is an instructive comparison in \cite{Trimillos1992} of the performance of pasyon and of wayang kulit.} 

The pasyon was sung in \textit{punto},\footnote{Literally, \enquote*{accent.}} a pattern of chant which corresponded to the character being sung. Christ was sung in a slow and meek manner, Mary in \textit{tagulaylay}, a mournful singsong chant residual from the performance of pre-Hispanic epics.

Doreen Fernandez writes, \enquote{It is logical to assume that from chanting the pasyon aloud, some towns progressed to assigning parts, then to adding costumes, and finally to having the parts acted out in costume.}\footcite[16]{Fernandez1996} \textit{Sinakulo }originated from pabasa. Sinakulo was the extended dramatic performance of the pasyon with actors, costumes, marches and special effects.

The staging of sinakulo required a large budget and was sponsored by the wealthiest families in a town. The dramatized pasyon opposed the \textit{banal}, the holy, slow of speech and movement, hands folded and eyes downcast in meekness and resignation, to the \textit{hudyo},\footnote{All villains in the pasyon, including the Roman soldiers, were known as \textit{hudyo}.} the Jews, who pranced about the stage, gloating, boasting, and were the entertainment of the performance. Innovation in performance was strongly discouraged and the holier the character represented, the stricter was the adherence to text and tradition. Only the hudyo, for whom variation in acting and changes in dialogue were not looked upon as blasphemous, engaged in comic behavior and innovation. To be holy was to accept suffering without complaint; it was to hold unswervingly to the script which God predestined for you.

\subsection{A Linguistically Specific and Class Universal Idiom}

The idiom which Ileto found in the pasyon was linguistically specific and crossed class boundaries. Both of these facts present serious problems for the arguments of \textit{Pasyon and Revolution}.

Rene Javellana compiled an excellent bibliography of pasyon texts and used this bibliography to construct a genealogy of the translation of the pasyon. This genealogy neatly captures the problem of the linguistic specificity of the pasyon.\footcite{Javellana1983} The pasyon was not translated into Pangasinan until 1855; Bikolano, 1867; Kapampangan, 1876; Ilokano, 1889; Hiligaynon, 1892; and Samare\~no, not until 1916. At least some of these non-Tagalog pasyon were not performed. Vicente Barrantes, a colonial observer commissioned by the Spanish government, wrote that \enquote{in Ilocos it is not the passion but the \textit{Lamentations of Jeremias} that is chanted during Lent. The former is not chanted but read, and that privately.}\footnote{Vicente Barrantes as quoted in \cite[165]{DelaCosta1965}. Barrantes is a problematic source for information on Philippine theater, as Rizal’s scathing response to him makes clear. The facts regarding the performance of the pasyon in Ilocos, however, are accurate.}  Thus, in Ilocos, the pasyon was a very recent introduction, which was not publicly performed, but privately read. It seems likely that the new pasyon librettos were purchased and read largely by the elite. 

We see that many regions vital to the progress of the Philippine revolution had only had the pasyon translated into their language a decade or two before the uprising. There was an additional lag between the translation of the text and its adoption in public performance. In the case of Ilocos but seven years lay between translation of the pasyon and the outbreak of the revolution. It strains credulity to assume that Ilokanos developed a deep seated pasyon idiom in this time which enabled them to conceive of a pattern to universal history and their role within it. And yet, the revolution in Ilocos was fierce, long lasting, and founded upon peasant and working class participation. This is true also of Samar which did not even receive a pasyon translation until long after the revolution ended. That the \enquote*{masses} responded to the revolution in a similar fashion in both Tagalog and non-Tagalog regions would suggest that the pasyon explanation is, at least partially, invalid. 

While the pasyon idiom did not cross linguistic boundaries, it was shared by Tagalog speakers of all classes. The elite participated in the performance of the pasyon and the creation of the pasyon idiom. It would have informed their understanding of the revolution just as much as that of the masses. Both the pabasa and sinakulo were events which crossed class boundaries. Their performance arena was a shared space in which landlords and tenants met, not as equals but as hierarchically ranked participants who spoke a common language, that of the pasyon.\footnote{On the role of Holy Week traditions in reinforcing hierarchy, see \cite{Venida1996}.}

Thus, the pasyon idiom which Ileto discovers cannot speak to the consciousness of specific class groups, not even to that of the amorphous \enquote*{masses.} It was a universal idiom. Rizal understood it as fluently as an \enquote{unlettered peasant.}

Vicente Barrantes, in his 1889 work \textit{El Teatro Tagalo}, criticized what he saw as the purely derivative nature of Philippine theater, in all its forms: pasyon, sinakulo, awit, and komedya. All Philippine theater, he claimed, was an imitation of Spanish literature, and a poor imitation at that. Jos\'e Rizal wrote a fiercely sarcastic response from Barcelona on June 15, 1889. He addressed the topic of the pasyon and awit.
\begin{quote}
\begin{otherlanguage}{spanish}
Por pobres y rudas que ellas pudieran ser; por infantiles, rid\'iculas y mezquinas que las tenga V. E., conservan sin embargo para mi mucha poes\'ia y cierta aureola de pureza que V.E no podr\'ia comprender. Los primeros cantos, los primeros sainetes, el primer drama que vi in me ni\~nez y que duro tres noches, dejando en me alma un recuerdo indeleble, a pesar de su rudeza e ineptitud, estaban en tagalo. Son, Excelent\'isimo Señor, como una fiesta intima de familia, de una familia pobre: el nombre de V. E que es de raza superior, la profanar\'ia y la quitar\'ia todo su encanto.\end{otherlanguage}\footcite[196; unless otherwise noted, all translations are my own]{Rizal1931}

As poor and rude as they maybe; infantile, ridiculous, and mixed compared to those works which belong to Your Excellency, they retain for me, however, great poetry and a certain halo of purity which Your Excellency could not comprehend. The first songs, the first sainetes, the first drama which I watched in my childhood, and which lasted for three nights, left in my soul an indelible memory, for in spite of their rudeness and ineptitude, they were in Tagalog. They are, Exalted Sir, like an intimate fiesta of a poor family: the name of Your Excellency, which is of a superior race, would profane and remove all of their enchantment.
\end{quote}

We see in this passage the intimate formative significance that both pasyon and awit had for the ilustrado exemplar, Rizal. Not only were the ilustrados present during the performance of pasyon, it was an integral aspect of their cultural experience.\footnote{Ileto pointed out that there were three Tagalog versions of the pasyon available in the nineteenth century, two quite polished and one, the Pasyon Pilapil, rough and of a poorer literary quality. He chose to work exclusively with the Pasyon Pilapil in \textit{Pasyon and Revolution} because of its \enquote{popularity among rural folk.} Might this be the class distinction of the pasyon, the poor and uneducated were familiar with a different version? Again no. Rizal, writing to Mariano Ponce from Paris in March, 1889, composed a short list of Tagalog luminaries which comprised three names: \enquote{Pilapil, Pelaez, Burgos.} \parencite[154]{Rizal1931}. All three were religious figures. Pelaez and Burgos were associated with the reform of the clergy. Burgos was executed after the failed Cavite mutiny and became identified as a nationalist martyr. Pilapil was associated with the pasyon.}

\textit{Pasyon and Revolution} examined the text of the pasyon sans performance. In performance the pasyon was a shared event which reinforced hierarchy and privilege. Rather than a unique window into lower class categories of perception, the pasyon was in fact one of the very few truly cross class idioms in nineteenth century Tagalog society. 

But what of awit?

\subsection{Awit as Performance}

The soldiers of Adelantado Miguel Lopez de Legazpi in the late sixteenth century are believed to have been the first to bring from Mexico the metrical romances of chivalry which were popular in their day. A continued trade in metrical romances flourished in the seventeenth and eighteenth centuries via the annual Galleon trade.\footnote{Irving Leonard studied the peregrinations of the metrical romances in his two classic works, \cite{Leonard1933}; and \cite{Leonard1949}.} These imported metrical romances had either dodecasyllabic or octosyllabic structure, and assonant verses.  

Metrical romances were eventually translated and became awit and \textit{corrido}.\footcite[5]{Fernandez1996} Corrido are octosyllabic poems, \enquote{which might be sung to the tune of the pasion,} while awit \enquote{are dodecasyllabic narratives sung in an elegiac and pleading manner.}\footcite[52]{Lumbera1986} The monorhymed quatrains of awit are called \textit{plosa}. There is a caesura after every sixth syllable. Every two lines complete a clause, and every four constitute a sentence. Prosody is exact and uniform. These metrical romances were originally propagated orally and were intended to be sung; awit simply means song in Tagalog.\footcite[4]{Castro1985}

Awit came to be performed in the eighteenth century as \textit{komedya}, dramas depicting the conflict between Christians and Muslims, who were derogated moros. Komedya became the centerpiece of nearly every town fiesta. Originally written by a folk poet in the town or barrio, in the nineteenth century the komedya became a more urbane, polished, and sophisticated form, written by poets such as Huseng Sisiw and Francisco Baltazar. Fernandez writes, \enquote{The years between 1820 and 1896 have been called the period of \enquote*{first flowering,} during which \enquote*{cosmopolitanization, urbanization and Christianization} merged to institutionalize the komedya.}\footcite[61]{Fernandez1996}

This first flowering, Resil Mojares argues, represented the culmination of the shift from \enquote{oral to written text, and from a living audience to a reading public,} it was \enquote{also a geographical shift to a proto-urban complex of school and print-shop.}\footcite[68]{Mojares1983} This shift was driven by the rise of an urban reading public, for whom the possession of awit chapbooks, or \textit{libros caballerias}, with their highly stylized poetry on foreign subject matter, was a mark of class and distinction. The chapbook became an important commodity in the self-construction of the hispanized Chinese mestizo.

Staged as komedya, these urbane awit became the viewing fare of the Manila working class and of rural laborers and peasants. Like sinakulo, komedya required a substantial budget to be performed. It thus required sponsors, patrons drawn from the town elite who would fund the production and receive in return honor and recognition. Each performance would open with a \textit{loa}, a long poem dedicated to the patron saint of the festival at which the komedya was performed and honoring the elite guests in the audience.

The staged komedya was a lengthy affair, performed in segments every evening over the space of three to five days. In Manila there were permanent theaters; Bonifacio was an actor in one of them, the \textit{Teatro Porvenir}. In the provinces the stage was a temporary construction. In either case, however, the stage was constructed according to a standard design. The fa\c{c}ade of a palace served as the backdrop, divided down the middle into two colors, which separated the Moro and Christian kingdoms. Spartan props indicated scene changes. The addition of chairs would create a palace court or potted plants a forest. Across the stage would march the actors. Marching played a central role in every production; different characters and different events called for different styles of marching, but no one ever walked. The marches were given folk Spanish names in the script: \textit{regal}, \textit{paso doble}, \textit{paseo}, \textit{karansa}, \textit{batalya}.\footcite[67]{Fernandez1996}

This script, however, was not available to the general public; it was not even available to the performers. It was called \textit{orihinal} and was hand copied by \textit{escribientes}. There was only one copy of the script in an entire performance and it belonged to the director, who had absolute control over all theatrical goings on. The actors did not memorize the verses of a komedya. They were fed their lines during performance by the \textit{apuntador}, who held the director’s script, and, from a hidden location, read each stanza to the actors, who would declaim the lines in \textit{dicho}, a singsong lilt designed to be heard by large audiences without the benefit of amplification.\footnote{Vicente Rafael builds a significant section of the thesis of his book, \cite{Rafael2005}, on the importance of untranslated bits of Castilian in komedya. The audience’s encounter with these untranslated foreign words was vital, he argues, for their imagining of the nation. Without examining the logic of this argument, it is worth noting that these untranslated Castilian words were stage directions which would only have been visible in the one copy of the script which sole property of the director. How the audience encountered these words remains a mystery.} 

Komedya were punctuated by \textit{sainete}, comic skits performed during breaks. A favorite character in the komedya was the jester, or \textit{pusong}. Unlike the other characters in the komedya who were not allowed to deviate even slightly from the script, the pusong could ad lib freely. He was allowed to make topical jests and political commentary. Such commentary, of course, was not preserved in the text of any komedya for historical analysis. The script of the komedya, with its stylization, authority, and structure, reinforced hierarchy and colonial values. In performance, the role of the pusong could often be subversive. We cannot, however, uncover this history through an analysis of the text of awit.

Fray Joaquin Martinez de Zuniga, who arrived in the Philippines in the late eighteenth century, noted that the native komedya tended to \enquote{satisfy the sight rather than the sense of hearing.}\footcite[9]{Fernandez1996} In particular, komedya were famous for their spectacular special effects. The audience delighted in fireworks set off on stage, characters lifted into the air with cords, and the brilliant marching patterns of the actors. As with the pasyon, the audience of komedya ate during performance, moved around, and talked loudly. They came and went freely and heckled actors who had trouble with their delivery.\footnote{Doreen Fernandez writes that an audience member might \enquote{return home briefly to cook dinner, during the play. When she returns, she will not really have missed a major part of the story or skipped a beat of the feeling, not only because the plot is episodic and references are repetitive, but also because it is assumed, predictable, and she hardly needs the actual performance to unfold the story for herself.} \parencite[177]{Fernandez1996}.}

\subsection{Bernardo Carpio, awit}

The awit which Ileto examined in detail was the \textit{Historia Famosa ni Bernardo Carpio}. The story of Bernardo Carpio was based upon the sixteenth century Spanish romances of Lope de Vega, in particular \textit{Las Mocedades de Bernardo del Carpio} and \textit{El Casamiento en la muerte}.\footcite[9]{Castro1985} The oldest extant copy of the \textit{Historia Famosa} dates to 1860. Subsequent editions have no changes in them. The more widely available 1919 edition printed by J. Martinez was a verbatim reproduction of the 1860 text. This clearly indicates that there were not multiple versions of the Carpio awit in circulation nor was it a text based upon a pre-existing oral tradition. The \textit{Historia Famosa} is an excellent example of the urbane, sophisticated compositions that came from the school of Huseng Sisiw and Francisco Baltazar in the mid-nineteenth century. 

Damiana Eugenio in her work, \textit{Awit and Corrido}\footnote{\cite{Eugenio1987}. This work was originally Eugenio's dissertation at UCLA in 1965.}, neatly summarized the \textit{Historia Famosa ni Bernardo Carpio}. Her summary, which \textit{Pasyon and Revolution} used almost word for word, collapsed together three distinct parts of the story which it would have been analytically useful to treat separately. The first is the narrative, based upon the Spanish text of Lope de Vega which relates the struggles of Bernardo both against the treacherous usurper who has betrayed his father, and against the moros. The second is the conclusion to the awit, which Eugenio treated as a localization and appendix. Bernardo Carpio, having conquered the moros, goes off to fight against \enquote*{idolaters,} worshippers of \textit{anito} or spirits. He sees lightning strike and two mountains colliding against each other and he plunges in between them, with sword drawn, and the mountains close after him. Finally, Eugenio summarized a \enquote*{legend} which told how Bernardo Carpio was imprisoned in a cave in San Mateo but would soon be freed and would liberate the oppressed of the Philippines.

There are two problems with the presentation of the \textit{Historia Famosa ni Bernardo Carpio} in both Eugenio's work and in \textit{Pasyon and Revolution}. First, they drew a sharp line between the story of Carpio's struggle against the moros and that of his subsequent struggle against idolatry. The latter section was treated as the localization of a foreign narrative, the \enquote{appropriation by the Tagalogs of a Spanish hero.}\footcite[9]{Ileto1998} Ileto argued that it is \enquote{curiously reminiscent of Colorum rituals.} (101) Many scholars, following Ileto, have gone so far as to treat this section of the awit as coded anti-colonialism; Christianity is seen as imprisoning pre-Hispanic beliefs. What these analyses overlook is the seamless connection between the two Carpio narratives, that of his struggle against the moros and of his subsequent struggle against idolaters. They are a single composition, written in the polished and sophisticated style of nineteenth century urban poetry. That which they treated as a \enquote*{localization} was in truth the extension of Spanish proselytization from the Moros to irredentist native beliefs. The colliding mountains, \textit{nag-uumpugang bato} (literally \enquote*{colliding rocks}), were a common feature of this belief system. Carpio travels here to fight the anito worshippers. He is not trapped at the end of the awit; rather, he enters the mountains to fight against residual pre-Hispanic beliefs. Eugenio and Ileto considered Carpio to be trapped because they read the awit in light of the separate legend.

The second problem in the treatment of this story was the incorporation of the Carpio legend as the conclusion to the awit. As we shall examine in detail below the legend was an entirely separate item, which emerged not from urban poets but from popular culture in opposition to the conclusion of the awit. It was never part of the text of the awit, nor was it ever performed in verse or circulated in written form. Legend is an entirely separate genre from folk poems, and the \textit{Historia Famosa ni Bernardo Carpio} was never even a folk poem; it was an elite textual composition. That \textit{Pasyon and Revolution} asserted \enquote{our uncertainty as to whether or not it appeared in the published awit,} (101-2, fn. 39) clearly indicates a failure to draw a basic distinction in genre. Much can be gained by treating the legend as a separate entity and examining its history without allowing the \textit{Historia Famosa} to color our expectations.

Ileto treated the characters in \textit{Historia Famosa} as metaphors. Carpio learns of the treachery of his stepfather and the imprisonment of his true father. From this story, Ileto argued, came ideas of breaking with false, and of liberating true, parents. These ideas were used by Bonifacio to help break the masses away from their debt of gratitude (utang na lo\'ob) to Mother Spain. \textit{Pasyon and Revolution} asserted, without any attempt at substantiation, that the masses felt indebted to Spain and conceived of their colonizers as their mother. The history of \textit{remontado} migration, of entire populations fleeing to the mountains, which we shall examine later on, belies the existence of this debt of gratitude. If Bonifacio truly was appealing to the masses using metaphors derived from Bernardo Carpio, and there is no evidence given for this, then his appeal would have fallen upon deaf ears.

\subsection{Chapbooks and class}

A glance at the back matter of the chapbooks in which late nineteenth century awit were published is instructive. These chapbooks were printed and sold in Manila by J. Martinez and are the source used by \textit{Pasyon and Revolution} for its analysis of awit. At the back of several of the awit chapbooks one finds a page entitled \textit{Salitaan sa Panyo}, \enquote*{speaking with handkerchiefs.} The page details a range of romantic messages that can be communicated by gestures with a handkerchief. \enquote{Ihapl\'os sa caliu\'ang cam\'ay: Icao ay quinapopootan co}/Wipe across the left hand: I despise you. \enquote{Ticlopin ang manga dulo: Hintain mo aco.}/Fold the ends: Wait for me. \enquote{Pilipitin nang camay na canan: May ibang iniibig aco}/Twist with the right hand: I love someone else. Other gestures with the handkerchief communicated: I have a fianc\'e, I am married, I am yours.\footcite{Buhay1916}  In like manner, the back page of another chapbook has the title \textit{Salitaan sa pamaypay}/\enquote*{Speaking with a fan} in its back matter, where we learn that to abruptly close the fan signified loathing, while to dangle the fan from the right hand was to indicate romantic availability.\footcite{Calaguim1923} The fan in question is not the large woven \textit{anahaw} fans of the lower classes, but the delicate folding \textit{abanico} fans of the elite. The back matter of the chapbooks makes very clear who was the audience for printed awit. It was the mestizo elite who, as they increased their wealth and power in the course of the nineteenth century, hispanized themselves, attempting to erase their indio and Chinese origins by the acquisition of artifacts, accents, behavior and culture from the Spanish metropole.

Nineteenth century awit was not the reading material of the peasantry or of the urban working class; they accessed these awit as komedya, that is, in performance. \textit{Pasyon and Revolution} found the wrong idiom because it read the wrong sources in the wrong way. 
