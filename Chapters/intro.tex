The publication of Reynaldo Ileto's \textit{Pasyon and Revolution}\footcite{Ileto1979} in 1979 produced a seachange in Philippine historiography. Ileto's work shifted the focus of historical research from the writings and actions of individual members of the elite to the perceptions and revolutionary participation of the lower classes. All subsequent research in Philippine history has been written in the light of Pasyon and Revolution. Reference to Ileto's conclusions is de rigueur for a field of studies whose subject matter ranges from the pre-colonial structure of the barangay to the economic policies of the Marcos regime. Benedict Anderson expressed the consensus of academic opinion when he wrote that \enquote{Ileto's masterly Pasyon and Revolution \ldots\ is unquestionably the most profound and searching book on late nineteenth century Philippine history.}\footcite[199 fn. 19]{Anderson1998} Despite the preeminence of \textit{Pasyon and Revolution} in Philippine studies, no one has written a comprehensive examination of the premises, source material, and conclusions of Ileto's work. This paper aims to fill this gap.

The first section of this paper examines the argument of \textit{Pasyon and Revolution} in detail. I find that Ileto's project of reconstructing the ways in which the lower classes of the Philippines in the late nineteenth century perceived the world and their role within it failed to achieve its goal for several reasons. Ileto never clearly defined what class or classes constitute his amorphous analytical category \enquote*{the masses.} He ignored how the source material which he studied was accessed through performance. As a result, Ileto read his sources as texts, in an elite manner, and reconstructed categories of perception with no demonstrable relationship to peasant or working class consciousness.

The second section aims to carry forward Ileto's project in the light of this critique. I study the legend of Bernardo Carpio in detail to demonstrate than when read with an attention to the significance derived from its performance, we arrive at a very different understanding of lower class consciousness than that which Ileto found. Rather than a counter-rational expression of peasant millenarianism, the legend was the \enquote*{hidden transcript} of subversive historical memory. It celebrated the history of social banditry in the region.

I argue in the third section that consciousness and perception, however carefully reconstructed, cannot in themselves explain dramatic historical events such as the Philippine revolution. To understand the causes of the Revolution and to account for the participation of the lower classes in it, we must give explanatory primacy to objective historical events and to the changes which occurred in the relations of production in the nineteenth century Philippines. These changes shaped consciousness and transformed the ways in which people perceived the world.
