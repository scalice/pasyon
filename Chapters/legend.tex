\section{Legend}
\subsection{Reconstructing perception using the Carpio legend}

If the approach to awit and pasyon in Pasyon and Revolution is deeply flawed and fails to achieve its goal of understanding the consciousness of the masses, where can we look if we wish to find a source which can achieve this goal? In the legend of Bernardo Carpio, Ileto located a source with great potential for analyzing actual lower class categories of perception. How did he read this legend?

As mentioned above, \textit{Pasyon and Revolution}, like all other scholarly work written on the subject, treats the legend and the awit as intimately connected; the legend is seen as the popular continuation of the \textit{Historia Famosa}. This approach extends to the legend the elite, textual hermeneutic employed in reading the awit. It also undermines recognition of the legend as a \enquote*{hidden transcript} of opposition to the urban elite who produced, and read, the awit.\footnote{On the idea of hidden transcripts, see \cite{Scott1990}.}

Like other scholars, Ileto not only conflated awit and legend, but he also combined multiple separate legends and variations of legends into an admixture from which little historical insight can be gained. We must disambiguate and analyze the various sources of the Carpio legend, situate them in their original contexts, and recreate how they would have been performed.

To be clear, the narrative of Bernardo Carpio chasing lightning into the colliding mountains is not a legend. This is part of an urban literary tradition, and was an integral aspect of the original 1860 composition. It represented an attempt to proselytize irredentist native beliefs, which were identified with the nag-uumpugang bato, the colliding rocks. The nag-uumpugang bato, two sheer cliff faces separated by a narrow canyon, are a comparatively common geographic feature in the karst topography of the Sierra Madre massif. They feature prominently in many legends and would thus have been identified by the author of the awit with traditional native beliefs.

The earliest version of the legend which I have been able to locate tells of an \enquote*{old man in the cave.} Ileto and others treat later fragments of this story as part of the Carpio liberator legend. It is a separate legend entirely. Gironiere's \textit{Twenty Years in the Philippines} is a source to which we shall return in much detail. It contains an appendix written in English in 1853 by the British explorer H. Hamilton Lindsay. In this appendix Lindsay told of his journey with Gironiere to the cave of San Mateo. He concluded his account by summarizing a legend. No previous scholar has drawn attention to this text, so I shall quote it in its entirety:

\begin{quote}
They have a curious legend respecting the cavern, which has a singular resemblance to the German tale of the \enquote{Three Brothers,} in the Hartz Mountains.\\
An Indian one day entered the cave to catch bats, with the wings of which they compound some sort of medicine. On arriving at the stream of water he saw a venerable old man on the other side, who offered his hand to help him across the stream. The Indian was rather shy of his new acquaintance, and held out the end of his stick, which the old man took, and it instantly turned into charcoal. Upon this the Indian became anxious to return, and thanking the old man for his politeness, told him he did not mean to go any further that day.\\
The old man then offered him three stones, and, to remove any fear of their burning his fingers, deposited them in the stream. The Indian took them, and retreated as quick as he could, without looking behind him; and, on examining the stones at the mouth of the cave, to his surprise he found them to be three masses of pure gold. The story did not go any further, as to what use he made of his riches. The old Indian who told me this story said it happened long before the arrival of the Spaniards.\footcite[353-4]{Lindsay1854}	 
\end{quote}

Lindsay would have heard this legend through an interpreter. We do not have the actual text of the legend, evidence of the texture of its performance, or the context in which it was traditionally performed. What we have is a legend summary. It is nonetheless quite useful. It will allow us to separate the various elements which later became identified with the Carpio legend. As Lindsay's account was published seven years before Bernardo Carpio entered Philippine literature in the 1860 \textit{Historia Famosa ni Bernardo Carpio}, we can safely say that the original San Mateo cave legend had nothing to do with him.

The Carpio liberator legend was first summarized by Jose Rizal in his second novel, \textit{El Filibusterismo}, which he published in 1891. The fifth chapter, \textit{A Cochero's Christmas Eve}, tells of a calesa driver, a \textit{cochero}, who has been detained by the \textit{guard\'ia civil} because he was missing his \textit{cedula}, the obligatory identification card of Spanish colonialism. As he is driving his passenger Basilio to the town of San Diego, they encounter a Christmas procession. The cochero sees the Three Kings in the procession,

\begin{quote}
And, observing that the black was wearing a crown and was a king like the other two Spaniards, he naturally thought of the King of the \textit{Indios} and sighed.\\
\enquote{Do you know, Se\~nor,} he asked Basilio respectfully, \enquote{if the right foot is free by now?}\\
Basilio repeated the question.\\
\enquote{The right foot? Whose?}\\
\enquote{The King's!} answered the \textit{cochero}in a low voice with much mystery.\\
\enquote{Which King?}\\
\enquote{Our King, the King of the Indios\ldots\ }\\
Basilio smiled and shrugged his shoulders.\\
The \textit{cochero} sighed again. The \textit{Indios} in the countryside treasure a legend that their king, imprisoned and chained in the cave of San Mateo, will one day come to deliver them from oppression. Every hundred years he breaks one of his chains and he already has his hands and his left foot loose; only the right foot remains chained. This king causes earthquakes and tremors when he struggles or is agitated. He is so strong that one can shake his hand only by holding out a bone, which upon contact with him is reduced to powder. For no explainable reason, the natives call him King Bernardo, perhaps confusing him with Bernardo Carpio.\\ 
\enquote{When the right foot is free,} murmured the \textit{cochero}, letting out a sigh, \enquote{I will give him my horses. I will place myself at his service and die for him\ldots\ He will free us from the civiles.}\footcite[36]{Rizal2007}
\end{quote}

Here we find preserved in Rizal's work a legend about an imprisoned liberator in the \enquote*{cave of San Mateo.} Aspects of the legend derive from the Old Man in the Cave legend, namely holding out the bone which upon contact is \enquote*{reduced to powder,} which corresponds to the stick turning to charcoal in Lindsay's version of the legend. We are still dealing with a legend summary, however; we do not have the actual text of the legend. Rizal accurately placed the legend on the lips of a member of the working class, a cochero, the driver of a horse-drawn \textit{calesa}.

Claudio Miranda, in a 1911 work on Philippine customs, provides us with additional insight into the legend.

\begin{quote}
Popular credulity has gone so far as to hope for the liberation of Bernardo del Carpio, one of the fantastic characters of Tagalog legend, imprisoned, according to the imagination of the commoners, between the two enormous rocks of Biaknabat\'o, so that he might exterminate the hunters who defend the Spanish forces. Nothing more is lacking but to free a single foot (\textit{paa na lamang ang culang}) in order to escape – they assure us – and when he is free, the war will be over, for Bernardo del Carpio can do anything.\footnote{\begin{otherlanguage}{spanish} \enquote{A tanto hab\'ia llegado la credulidad popular, que, hasta esperaban la liberaci\'on de Bernardo del Carpio, uno de los personajes fant\'asticos de una leyenda tagala, preso, seg\'un la imaginaci\'on del vulgo, entre las dos enormes rocas de Biaknabat\'o, para exterminar \'a los cazadores que defend\'ian las avanzadas espa\~nolas. No le falta m\'as que soltarse un solo pie (paa na lamang ang culang) para escaparse -- aseguraban -- y cuando est\'e libre, la guerra habr\'a terminado, porque Bernardo del Carpio todo lo puede.} \end{otherlanguage} \parencite[62-3]{Miranda1911}.} 
\end{quote}

Miranda's version of the legend is problematic on many levels. He was at a greater historical remove from the context of the performance of the legend. He was a much less sensitive observer of Philippine society than Jos\'e Rizal. He refers to Carpio as Bernardo \textit{del} Carpio, the name used in the Spanish version of Lope de Vega; in Philippine literature he is known simply as Bernardo Carpio, the locative has become a surname.

Miranda places Carpio not at San Mateo, but at Biaknabat\'o, another location famed for its nag-uumpugang bato.\footnote{Biaknabat\'o achieved infamy during the Revolution, after the death of Bonifacio, for the conclusion of a treaty of peace between the Spanish government and the forces of Emilio Aguinaldo. Aguinaldo and his coterie of leaders received 800,000 Mexican dollars for the cessation of hostilities and went into exile in Hong Kong. The rank-and-file continued the revolution against Spain in their absence.} It seems unlikely that Miranda is recording a geographical variant of the legend; rather, he is simply erring in his summary. He does, however, provide us with an invaluable fragment of an actual performance of the legend: \enquote{Paa na lamang ang kulang/only the foot is lacking.}

A version of the legend recorded in 1917 has Rizal visiting the old man in the cave of San Mateo who is revealed to be Bernardo Carpio. Rizal extends a bone to Carpio and it crumbles to dust when he touches it. Rizal returns and informs others that Carpio has only one foot still chained.\footcite[41]{Ileto1982} In a version of the legend documented in 1940, Carpio is no longer chained, but imprisoned by God \enquote{for his sins} and is lying among the dead. A bone is extended to him and he crumbles it to dust. He tells his visitor to devoutly say \enquote{Christum} to ward off danger, adding that he would soon rise to save the \enquote*{oppressed people,} in keeping with the reasons of Almighty God (100).

By the time these last two legends were summarized, Bernardo Carpio had become a residual tradition. Idiolect had come to dominate performance. Elements persist: the bone, San Mateo, etc., but the legend was no longer anchored in a community. These later summaries are of dubious value for recreating legend performance at the time of the Philippine revolution. 

We see a dynamic and evolving legend with multiple variants. Ileto, like all other scholars on this subject, collapsed these variants together as a single narrative. Having conflated these variants, what use does Ileto make of the legend? He connected the legend with Bonifacio, telling of Bonifacio's journey in 1895 to the cave of San Mateo. Bonifacio and his katipunero associates traveled to the cave during Holy Week. \enquote{Could it be merely coincidental \ldots\ that the group chose the Holy Week of April, from Holy Tuesday to Holy Saturday, to make the climb?} (99) \textit{Pasyon and Revolution} inquired. The question implies that the trek of the Katipunan leaders should be linked with religious journeys and pilgrimages, and the collection of anting-anting. But Bonifacio and his companions traveled during Holy Week for a more mundane reason, one which any worker would understand. During Holy Week all business shuts down. This would have been the only opportunity for a group of eight wage laborers to travel together and to do so without raising suspicion.\footnote{A British businessman, living in Manila at the time, wrote: \enquote{To-day is the beginning of Easter Week, nearly all of whose days are holidays or holy days. This is one of the closest-observed seasons of the year, and on next Thursday and Friday, if you will believe it, no carriages are allowed to appear in the streets either of Manila or the other cities \ldots\ It seems the proper thing to do to make arrangements with some of the English colony \textins{i.e., the other English residents of Manila} to take a trip off into the mountains\ldots\ } \parencite[58-9]{Stevens1898}.}

It does not matter that the leaders of the Katipunan traveled during Holy Week for purely pragmatic reasons, that there was no \enquote{pilgrimage,} that Bonifacio did not write on the wall -- what is important, according to Ileto, is how the \enquote*{masses} would have perceived the event. What do we learn in \textit{Pasyon and Revolution} from the Carpio legend, and Bonifacio's visit to the cave of San Mateo? Sadly, little. According to Ileto, the masses actually believe in existence of Bernardo Carpio. The masses inhabit \enquote{a society where King Bernardo Carpio was no less real than the Spanish governor-general.}\footcite[63]{Ileto1982} Bonifacio, by traveling to the cave, was perceived as identifying with this real king, he was seen as trying to awaken him. Bonifacio thus inspired the devotion of the masses. Ileto treated the legend of Bernardo Carpio as a counter-rational, messianic means of mobilizing dissent. 

To read legends as embodiments of the actual beliefs of the \enquote*{masses} is to read in a manner that is both elite and na\"ive.

How should we read legend?

\subsection{Legend as Performance}

To understand the significance of the Carpio legend we must do more than establish the meanings of the words and sayings it contained. We must seek the effect of the legend's performance in its historical social context.  

The performance of a legend, when addressed to a community familiar with it, brings to life an entire body of tradition. To grasp the legend's meaning we need to recreate the lost context of oral tradition which lurks behind the entexted or summarized utterance.  Tradition cannot be reduced to intertextuality, it is the entire nexus of ideas and allusions which a culture creates and upon which it thrives.

Oral traditions generally have a great deal of regional variation. For the legend genre in particular it is the geographical referents, the allusions to place, which most commonly vary as the legend spreads. It is striking that the summaries of the Carpio legend preserve the geographic specificity of the caves of San Mateo. Timothy Tangherlini summarized the scholarship on the legend genre in his article, \enquote{It happened not too far from here \ldots}

\begin{quote}
Legend, typically, is a short (mono-) episodic, traditional, highly ecotypified, historicized narrative performed in a conversational mode, reflecting on a psychological level a symbolic representation of folk belief and collective experiences and serving as a reaffirmation of commonly held values of the group to whose tradition it belongs.\footcite[385]{Tangherlini1990}
\end{quote}

The \enquote{high ecotypification} to which Tangherlini refers is geographic, this is why legends \enquote{happen not too far from here.} Why then did the Carpio legend resist localization away from San Mateo? The answer is that San Mateo was indispensable to the legend. Oral performance invokes the large and invisible body of tradition through the use of metonym, a part representing the whole. A particular fragment of tradition is insistently repeated in performance. When entexted these awkward repetitions are often smoothed over and erased to match the literary sensibilities of the reading audience. These repeated fragments serve as integers which, to an audience alive to the body of tradition being invoked, convey meanings far larger than the actual words suggest.

This metonymic indexicality allows the performer to communicate in a restricted code, one intelligible to others familiar with the code, but seemingly innocuous or nonsensical to those outside it. Thus, the phrase which we found in Miranda's work, \enquote{paa na lamang ang kulang/only the foot is lacking} would have served, for those familiar with it, to refer to the entire Carpio legend and its broader meanings; to outsiders, however, it would seem to be meaningless or simply an example of the superstitious credulity of the masses. There is thus a rupture in meaning when metonymic references are listened to outside of their intended register. By failing to pay heed to the register of performance and to the indexical role of certain elements in the legend, Ileto and Rizal arrived at the idea that the masses sincerely believed in the existence of an actual king.

Legend is the \enquote{reaffirmation of the commonly held values of the group;} the performance of legend is perlocutionary, it enacts community solidarity. This could be done in the presence of the ruling classes without fear of reprisal. The odd phrase \enquote{paa na lamang ang kulang/only the foot is lacking,} which Miranda gives us, would have served to invoke the entire Bernardo Carpio legend for those familiar with it, while leaving elite observers mystified.

To those attuned to the register of performance and to its metonymic function, each performance of an element of oral tradition serves not to create new meaning but rather to invoke meaning which was already immanent in the tradition. Around what aspects of tradition did the Bernardo Carpio legend strengthen community solidarity? What are the repeated metonymic elements of the legend? The pervasiveness of the Carpio legend throughout the Tagalog speaking provinces and its strong resistance to synchronic ecotypification at the time of the Philippine revolution both point to the geographic elements of legend being of central metonymic significance. What body of traditions would reference to San Mateo invoke? 

To anticipate results which I shall substantiate in detail: the Carpio legend was not a counter-rational messianic means of mobilizing dissent; it was a record of resistance. Through the geographical metonym of San Mateo, the Carpio legend preserved and celebrated the memory of social banditry.

\subsection{Pamitinan and Tapusi}

To make clear how the Carpio legend and San Mateo referred to social banditry, we must make the relationship between Pamitinan and Tapusi evident. No scholar has yet studied the relationship of these locations and so it will be necessary to go into some detail. Pamitinan is a mountain and is the location of the caves of San Mateo. Ileto often referred to this mountain as Tapusi. Why? What were these two places? 

Sixto de los Angeles, the president of the Provincial Board of Health in the province of Rizal, writing on October 27, 1902, analyzed the sources of the Manila's water supply. The water came for the mountains of Montalban. A parenthetical aside in his report is instructive.

\begin{quote}
The stream flowing toward Montalban is very small near its source but it receives the water from several branches in the various points where the river passes, some of which are larger than the principal stream, the more important being, from its origin, the following: Lumutan (the name comes from the fact that rain falls throughout the year and the trees are always green), Sare or \textit{Tapusi (popular name since immemorial times as an inaccessible den of ladrones)} Uyungan, Dumiri, Taladoy, Tayabasan, Bunbunan, Astampa, Kal, Kayrupa (where a larger stream enters), Kaykaro, and then the caves, distant about 3\( \tfrac{1}{2} \) miles from Montalban, at which point the river passes between two mountains, forming the caves. Many people think these caves are the origin of the river, but in fact only one small stream issues from one of the caves. The mountains here form a narrow defile with many large marble stones.\footnote{\cite[221]{Angeles1904}, emphasis added.} 
\end{quote}

The mountains forming a \enquote*{narrow defile} are the \enquote*{nag-uumpugang bato} of the Carpio legend. Montalban and San Mateo were adjacent towns and the caves were occasionally referred to as the caves of Montalban. In this paragraph we see that an important source of Manila's water, the San Mateo river, which rushes through the gorge at the foot of Pamitinan and Sasocsungan mountains, has its origins in a region named Lumutan and runs through Tapusi, which was \enquote{since immemorial times an inaccessible den of ladrones.} Ladrones were bandits, widely known as tulisanes. The cave of Bernardo Carpio is in Mount Pamitinan, which is on a spur of the Sierra Madre massif; this spur was referred to as the Mountains of San Mateo. It is the closest encroachment of the Sierra Madre mountains to Manila.\footnote{On Pamitinan, Sasocsungan and the cave, see \cite[s.v Pamitinan; s.v. Mateo (San)]{Buzeta1850}.}

\enquote*{Lumutan} was another name for the Limutan river valley; it is over forty kilometers from San Mateo, and was separated by uncharted mountainous terrain. How then did Tapusi come to be identified with Pamitinan, so that Ileto and other scholars would speak of Bonifacio's ascent of Mt. Tapusi?

\subsection{Gironiere, Dumas, history and fantasy}

One of the earliest and most important sources for this examination is Paul Proust de Gironiere's work. Gironiere's writings are prickly, problematic sources.  Of all the travel narratives written in the Philippines in the nineteenth century, his account was based on the most time spent there. Gironiere lived in the rural Philippines for twenty years from 1819 to 1839, the owner of a plantation on a Laguna peninsula known as Jalajala. His work is regarded as an excellent source on the cholera epidemic of 1820 and the massacre of the French residents, who were blamed by indios for the outbreak. The cholera riots provoked fears of revolution among the Spanish authorities in the wake of events in Mexico.\footcite[29]{DeBevoise1995} As Gironiere is a source of much unique information, it is necessary to investigate his credibility.

Gironiere claims in the preface to his work that he was inspired to write his own version of events when he read a \textit{feuilleton} by Alexandre Dumas P\`ere in \textit{Le Constitutionelle}. This feuilleton was subsequently published as \textit{Les Mille et un Fantomes}. Dumas' novel was a m\'elange of material: several lengthy and unconnected narratives, a memoir of one of Dumas' recently deceased friends, and a story entitled \textit{Les marriages de pere Olifus}. \textit{Les marriages} told of M. Olifus, who, pursued by his mermaid wife, journeys to Bidondo\footnote{Binondo, Manila's Chinatown.} (\textit{sic}) and meets a Chinese woman, Vanly Tching, whom he marries. He then travels to Halahala \textins{\textit{sic}} where he converses with M. de la Geronniere \textins{\textit{sic}}.\footnote{This story was subsequently published separately from \textit{Les Milles et un Fantomes} and all succeeding editions of \textit{Les Milles} lacked the story of M. Olifus. Thus Andrew Brown's delightful recent translation, \cite{Dumas2004}, with its ghastly ruminations on the persistence of consciousness in guillotined heads does not contain Olifus' narrative or the encounter with Gironiere. \textit{Les marriages} was translated into English and published as \cite{Dumas}}

In the late 1840's Gironiere had been holding forth in the salons of Nantes, regaling audiences with his tales of adventure in the Philippines and word of his stories reached the intellectually omnivorous Dumas. Stories of banditry were regarded as romantic and were wildly popular in Europe in the mid-nineteenth century\footnote{On this point see \cite{Hobsbawm2000}.}, and Gironiere told many bandit stories. Stumbling upon himself as a character in \textit{Le Constitutionelle}, Gironiere wrote to Dumas and offered to sell his own story for publication in Dumas' new journal, \textit{Le Mousquetaire}.

Europe had just been rocked by a series of working class revolutions and their bloody suppression by governments. A marked shift occured in Dumas' writing. An author who previously wrote romanticized yet trenchantly political stories, in historical settings which were but lightly fictionalized, Dumas now wrote a volume of fantasy with ghosts and mermaids and a journey to the exotic orient.

Dumas was well aware of the political reality from which his work was moving away. The opening chapter of \textit{Les Milles et un Fantomes} is masterful in its realistic depiction of the working class. The narrator leaves Paris and travels to Fontenay-aux-Roses. As he passes Grand-Montrouge he sees quarries, where men run in gigantic wheels, engaged in \enquote{squirrel-like labor,} raising stones from the depths. \enquote{\textins*{I}f he actually rose one step in height each time his foot rested on a strut, after twenty-three years he would have reached the moon.}

He compares the landscape to a Goya engraving and notes that these are the stones that built Paris. The land has an abyss beneath its seemingly beautiful scenery through which a man could fall. \enquote{The verdant earth that seems so alluring rests on nothing; you can, if you set your foot over one of these cracks, quite easily disappear.}

The populace of these galleries has a separate physiognomy and character. \enquote{You often hear of an accident: a prop has collapsed, a rope has snapped, a man has been crushed. On the surface of the earth, this is taken to be a misfortune: thirty feet under, it is known to be a crime \ldots\ The appearance of the quarrymen is in general sinister \ldots\ Whenever there's any civil commotion, it is rare that the men we have just been trying to depict do not get involved. When the shout goes up at the barri\`ere d'Enfer, \enquote{Here come the men from the Montrouge quarry!} the people living in nearby streets shake their heads and shut their doors.}\footcite[4-6]{Dumas2004}

From this realistic depiction of working class anger, Dumas turns to a story of ghosts, and a journey to the exotic east. The culmination of this journey into unreality is the encounter with \enquote*{Geronniere.}

The revolutions of 1848-49 saw the rise of realism in art; for Dumas, they marked a flight from reality. His later work served as the inspiration for Hoffmann's Nutcracker. There was nothing innocent in this literary choice by Dumas; he dedicated \textit{Les Milles et un Fantomes} to the Orleans dynasty.

Thus, Dumas made a shift in his writing to fantasy and Gironiere's salon fabrications provided the content of that fantasy for him. Gironiere's work told how he single-handedly stopped a war, visited cannibals and headhunters and ate brains with them, and, above all, how he constantly encountered, captured, and conversed with bandits.

While Gironiere was waiting for his stories to be published in \textit{Le Mousquetaire}\footnote{In which journal it was serialized in 1855.}, he tried getting them published elsewhere. He published his adventures as \textit{Vingt Ann\'ees} in 1853. He revised this slightly and republished it as \textit{L' Aventures d'un Breton} in 1855, adding an appendix.\footnote{Even this addition has occasional moments that would appear to be playful or fictionalized. In a list of Tagalog words and their French equivalents, Gironiere lists susu as saint (holy). Susu, depending on the accent, means either snail or breast.}

We have thus from the outset many reasons to be skeptical of Gironiere's writing. His account was the most widely read popular work on the Philippines in the nineteenth century.\footcite[56]{Sullivan1991} Many travelers noted in their own accounts the fictional nature of Gironiere's stories; some did so gently, others not so gently.

John Bowring, fourth governor of Hong Kong, wrote in 1859,
\begin{quote}
I can hardly pass over unnoticed M. de la Gironiere's romantic book, as it was the subject of frequent conversations in the Philippines. No doubt he has dwelt there twenty years; but in the experience of those who have lived there more than twice twenty I found little confirmation of the strange stories which are crowded into his strange volume \ldots\ M. de la Gironiere may have aspired to the honour of a Bernardin de St. Pierre or a Defoe, and have thought a few fanciful and tragic decorations would add to the interest of this personal drama. \enquote{All the world’s a stage,} and as a player thereon M. de la Gironiere perhaps felt himself authorized in the indulgence of some latitude of description, especially when his chosen \enquote{stage} was one meant to exhibit the wonders of travel.\footcite[101-2]{Bowring1859}
\end{quote}

Henry Ellis can be read on the subject with amusement. He begins his travels in the Philippines in deep admiration of Gironiere's work and is gradually disappointed on all counts. This disappointment is told in a tongue-in-cheek fashion. On journeying to Lagunita de Socol, he remarks, \enquote{Gironiere estimates this hill at 1,200 or 1,500 feet high; we thought, unanimously, that about 100 was nearer to the mark\ldots\ }\footcite[103]{Ellis1859} Ellis tells of traveling to Jalajala, Gironiere's former estate, of looking for \enquote{Tulisanies,} \textins{\textit{sic}} of his assessment of Gironiere's description of a kalabaw/carabao, and of desiring to eat brains, \enquote{a la Gironiere,} and is each time disappointed and yet still praises Gironiere's book. Cf. ia pp. 15, 88, 91-2, 94-7, 102-3, 191, 207.\\
Laurence Oliphant remarks wryly of Gironiere, whom he refers to as \enquote{that amusing but most audacious romancer,} \enquote{we trust, for the sake of La Gironiere's credit as a sportsman, that he displayed as much courage with his rifle as he certainly has with his pen.}\footcite[88-9]{Oliphant1860}

Finally, the German naturalist, Fedor Jagor, remarks in a footnote on Gironiere, \enquote{The raw materials of these adventures were supplied by a French planter, M. de la Gironiere, but their literary parent is avowedly Alexander Dumas.}\footcite[29; The original German work, \textit{Reisen in den Philippinen}, was published in 1873. No translator is cited for the 1875 English edition]{Jagor1875}

\subsection{Gironiere and \enquote*{Tapuzi}}

All of that said, Gironiere was a plantation owner in rural southern Luzon from 1819 to 1839. This is a period and a location for which we have few sources. Any accurate material which could be gleaned from his account would thus be unique.\footnote{Gironiere is not only an important source of historical information; he was an important literary influence. He wrote a small privately published work late in his life entitled, \cite{Gironiere1862}. It received no notice in the nineteenth century, but wound up as item 1184 in T.H. Pardo de Tavera's \textit{Biblioteca Filipina}. The ilustrado community in Madrid would thus have had access to this text. It tells of a journey to Majayjay, the site of the Cofrad\'ia de San Jose uprising, and of Gironiere's encounter with a bandit, with whom he has a lengthy discussion about legal and illegal means of changing society. The dialogue parallels the Ibarra-Elias dialogue of Rizal's \textit{Noli} very closely. The work was translated into English as \cite{Gironiere1983}. The dialogue with the bandit runs from pages 19 to 31.} We must, however, approach this material with extreme caution and deep hermeneutical suspicion if we are to recover anything of historical value. Our approach must discard as historically unreliable any portion of the text which glorifies Gironiere.\footnote{For instance, Gironiere told a story of shooting a monstrous crocodile. Many ridiculed the story as outlandish. The truth was a bit more complicated. There was an enormous saltwater crocodile shot when and where Gironiere claimed. The skull was sent to Harvard Museum of Natural History and is still displayed there. Gironiere, however, did not shoot the crocodile; an English visitor to the region did. See \cite{Barbour1924}.}

Gironiere wrote of an excursion to \enquote*{Tapuzi.} He introduced Tapuzi as \enquote{a place where bandits, when hotly pursued, were enabled to conceal themselves with impunity.}\footnote{\cite[113]{Gironiere1962}; all of the following is taken from this account pp. 113-7.} It was \enquote{situated in the mountains of Limutan. Limutan is a Tagalog word, signifying \enquote{altogether forgotten.}} The name Tapuzi, he stated, meant \enquote*{end of the world.}

Gironiere claimed that Tapuzi was formed in Limutan by bandits and men who had escaped from the galleys, who now \enquote{live in liberty, and govern themselves \ldots\ I have often heard this singular village mentioned, but I had never met anyone who visited it, or could give any positive details relative to it.}

In his story he travels with a guide, a former bandit, up a ravine which was defended from above by stones which could be pushed down upon intruders. An immense block of stone falls in front of them; it is a warning. They are then led by guide from Tapuzi to a village of sixty thatched huts. He meets with the \enquote*{matanda sa nayon}/village elder, leader of Tapuzi. When Gironiere identifies himself, the old man responds, \enquote{It is a long time since I heard you spoken of as an agent of the government for pursuing unfortunate men, but I have heard also that you fulfilled your mission with much kindness, and that often you were their protector, so be welcome.}

The \enquote*{Tapuzians} feed Gironiere \enquote{milled corn and kidney potatoes.} This is an accurate description of a diet which swidden agriculture in the Sierra Madre mountains would have allowed. Although the majority of \textit{kaingin} -- mountain or jungle fields cleared for planting -- are used for rice, occasionally corn is grown instead. The old man tells Gironiere, \enquote{Several years ago \ldots\ at a period I cannot recollect, some men came to live in Tapuzi. The peace and safety they enjoyed made others imitate their example \ldots\ } Tapuzi would thus have been populated by waves of remontado migration. This corresponds with our few other sources on the matter.

The old man tells Gironiere of the social and economic structure of the village. \enquote{Almost all is in common \ldots\ he who possesses anything gives to him who has nothing. Almost all our clothing is knitted and woven by our wives; the abaca \ldots\ from the forest supplies us the thread that is necessary; we do not know what money is, we do not require any. Here there is no ambition; each one is certain of not suffering from hunger. From time to time strangers come to visit us. If they are willing to submit to our laws, they remain with us; they have a fortnight of probation to go through before they decide. Our laws are lenient and indulgent.}

Based on this information, Gironiere describes Tapuzi as \enquote{a real, great phalanstery, composed of brothers, almost all worthy of the name \ldots\ On the other hand, what an example that was of free man not being able to live without choosing a chief, and bringing one another back to the practice of virtuous actions!} Gironiere is editorializing. His reference to phalansteries and thus to Fourier, is completely out of place. Hermeneutical suspicion dictates that we must throw out all of his material on the social structure of Tapuzi. The line about not knowing what money is, is particularly suspect; it is likely that the remontado population was actively engaged in trade.

The old man continues. Formerly \enquote*{Tapuzians} lived \enquote{like savages.} But the old man had restored Christian practices. \enquote{I \ldots\ put my people in mind that they were born Christians.} He officiated mass, celebrated marriages, and baptized infants. Information in Norman Owen's study of Bicol indicates that remontados would occasionally enter the villages to receive religious services.\footcite{Owen1984} We will discard Gironiere's claims regarding the old man functioning as the village priest.

Gironiere offers to inform the Archbishop of Manila that he might send a priest. The old man declines. \enquote{We should certainly be glad to have a minister of the Gospel here, but soon, under his influence, we should be subjected to the Spanish government. It would be requisite for us to have money to pay our contributions. Ambition would creep in among us, and from the freedom we now enjoy, we should gradually sink into a state of slavery, and should no longer be happy.} This seems again to be Gironiere editorializing.

None of the Tapuzian women, Gironiere observes, had ever been out of their village, and had scarcely ever left their huts. This statement is absurd.

Prior to Gironiere's departure the old man tells him a legend: \enquote{At a time when the Tapuzians were without religion, and lived as wild beasts, God punished them. Look at all the part of that mountain quite stripped of vegetation: one night, during a tremendous earthquake, that mountain split in two -- one part swallowed up the half of the village that then stood on the place where those enormous rocks are. A few hundred steps further on all would have been destroyed; there would no longer have existed a single person in Tapuzi; but a part of the population was not injured, and came and settled themselves where the village now is. Since then we pray to the Almighty, and live in a manner so as not to deserve so severe a chastisement as that experienced by the wretched victims of that awful night.}

We are at many degrees of remove from any original legend that Gironiere may have heard. All that we can say to be likely is that there was a legend associated with Tapusi which pertained to a mountain which was split in two. This correlates nicely with the many legends of nag-uumpugang bato. We thus see the legend of the origin of Tapusi associated with the same geographical feature which dominates the Carpio legend of San Mateo. We cannot however treat the text of the legend as it is found in Gironiere’s account seriously; it is a continuation of his editorializing. In the end, we must conclude that all of his conversation with the old man is suspect and should be discarded for purposes of historical reconstruction.

\begin{figure*}[p]
    \centering
	 \captionbox{Carte Topographique du Lac de Bay. \label{fig:tapuzi}}{
		\includegraphics[width=0.9\textwidth]{tapuzi.eps}
		}
\end{figure*}

\textit{Vingt Ann\'ees} was translated by the author and published in the United States in 1854 as \textit{Twenty years in the Philippines}. The English edition did not include a map. \textit{Vingt Ann\'ees}, however, did. (See \cref{fig:tapuzi}.) It is a beautiful, A4 sized fold-out map in the back of the book, and is unique in Philippine cartography.\footcite{Gironiere1853} The map clearly indicates the approximate location of Tapuzi, at considerable remove from San Mateo and Pamitinan, in the Limutan valley. Waterways are marked in blue, Jala-jala in pink. The waterway which enters Laguna de Bay at Tanay stretches up between Bosoboso and Tapuzi. Valle Tapuzi sits between two rivers which unite to its south and head eastward off the map. These rivers are not printed in blue, but are clear. At the upper left a winding river is labeled Valle de Lanatin, on the upper right, \enquote{Sabang del R\textsuperscript{o}. Limutan.} This fork never reaches the top of the page.  The river that they unite to form reads \enquote{R\textsuperscript{o}.  Gaudaboso aue desagua en el mar de Binangonan de Lampong.} To the right of this river: \enquote{Darangitan.} \textins{Daraitan}

This location for Tapuzi/Tapusi is historically accurate. It is borne out by a history of the parishes of the religious province of San Gregorio Magno written in 1865 by the discalced Franciscan friar F\'elix de Huerta. A paragraph buried within the 720 page tome states

\begin{quote}
\textsc{Limotan}\\ 
Some eight leagues distant from the mission of San Andres, to the north across an elevated spine of mountains, is the River Limotan and on its banks is a rancher\'ia \textins{small settlement}, which, was gathered by Francisco de Barajas, and made Christian by the signing of a pact on May 6, 1670, and on the next day May 7, in the said year, were baptized the first seven people of the said rancher\'ia. From the year of 1670 to that of 1675 the fervent zeal of the above mentioned Fr. Francisco de Barajas caused many more to join the mission, including the surrounding rancher\'ias named Tapusi, Asbat, Mamoyao, Macalia, Dadanbidig and Maquiriquiri, Bantas, and Binoagan.\\
This mission grew prosperously until the year 1700, at which time the government had intended to oblige the mission to pay tribute. All fled to the mountains, the mission was completely lost.\footnote{\begin{otherlanguage}{spanish}\enquote{\'A unas ocho leguas distante de la mision de S Andr\'es, h\'acia el N atravesando una elevada cordillera de montes, se halla el rio Limotan y en su márgen habia una rancher\'ia, la cual, convino con nuestro R.P. Fr. Francisco de Barajas, hacerse cristiana por escritura firmada el dia 6 de Mayo de 1670, y en efecto el siguiente dia 7 de Mayo, de dicho a\~no, fueron bautizadas la siete primeras personas de dicha rancher\'ia. Desde este a\~no de 1670 hasta el de 1675 fueron agreg\'andose \'a esta mision por el celo fervoroso del citado Fr Francisco de Barajas, las rancher\'ias circunvecinas denominadas Tapusi, Asbat, Mamoyao, Macalia, Dadanbidig, y Maquiriquiri, Bantas, y Binoagan.}\\ 
\enquote{Esta mision sigui\'o prósperamente hasta el a\~no de 1700, en cuya \'epoca habiendo intentado el Superior Gobierno obligarlos \'a pagar tributo, se huyeron todos al monte, perdi\'endose enteramente la mision.} \end{otherlanguage} \parencite[573]{Huerta1863}. 
}
\end{quote}

Here we see that prior to being an \enquote{inaccessible den of ladrones,} Tapusi was a rancher\'ia, a small pueblo. It became part of the Spanish mission of San Andres, in the \enquote*{Limotan} river valley, but the residents fled to the mountains as remontados in 1700 when forced to pay tribute. Gironiere's geography is accurate.

Gironiere's account pared back to its core of plausible historical details reveals a community of remontados, built up by waves of migration, engaged in subsistence corn agriculture, located in the Limutan River valley, with an origin legend based on the same geographical feature as the Bernardo Carpio legend: nag-uumpugang bato.

And what of the cave of San Mateo? 

\subsection{Tourist pilgrimages}

Bonifacio's trek to San Mateo could be situated within a history of \enquote*{pilgrimage,} but it would be a very different history from that which \textit{Pasyon and Revolution} suggested. There was an established tradition of European tourists traveling to the cave of San Mateo in the nineteenth century.

In a separate section of his book, Gironiere writes of traveling to see the cave of San Mateo. He tells of going between two \enquote{monster mountains \ldots\ equally alike and similar in height.}\footcite[128]{Gironiere1962} His story goes into great detail of the spelunking which he and Hamilton Lindsay undertook, through subterranean chambers and between enormous stalactites. He does not mention Tapuzi in the context of the cave of San Mateo, nor does he mention San Mateo in his journey to Tapuzi. At the time of his explorations the conflation of the two locations had not yet occurred.

Surveying the accounts written by foreigners visiting Luzon in the nineteenth century we almost always find a reference or two to the caves of San Mateo. It was a popular destination among the more bold adventurers to visit Manila.

The Scottish businessman, Robert MacMicking, wrote

\begin{quote}
Some miles beyond Mariquina, there is a most curious cave, of great extent, at the village of San Mateo, which is well worthy of a visit by the curious. Shortly after entering it, the height of the cavern rises to about fifty feet, although it varies continually, -- so much so, that at some places there is scarcely height enough for a man to sit upright \ldots\ The temperature within the cavern was 77\degree, and without, 86\degree, being a very considerable change, even in the cool of the evening, on coming out of it, just after sunset. I am afraid to give an estimate as to the extent of this immense cave; it requires, however, five or six hours to partially see its curiosities, and of course would take far more time to investigate it properly. The only living creatures met within it, \textins{\textit{sic}} appear to be bats, which are not very numerous.\footcite[107-8]{MacMicking1851}.
\end{quote}

Joseph Stevens travelled to the cave in May, 1894, less than a year before the Katipunan visited the cave during Holy Week, 1895. He wrote,

\begin{quote}
After a jolly good bath, and a few preparations, our party of four, with the two boys and two guides, started up a steep valley in among lofty mountains to the so-called caves of Montalvan.\textins{\textit{sic}} One of our guides was the principal of a village school, who held sway over a group of little Indian girls under a big mango-tree, and he shut up shop to join our expedition. In about two hours and a half our caravan reached the narrower defile that pierced two mountains which came down hobnobbing together like a great gate, grand and picturesque. From a large, quiet pool just beneath the gates, we climbed almost straight up the mouth of the stalactite caves that run no one knows how far into the mountains, starting at a point about two hundred feet above the river.\footcite[89-90]{Stevens1898}.
\end{quote}

It was not just foreign travelers, but business also which was going to the caves of San Mateo. One year prior to Stevens' journey, the San Pedro Mining Company petitioned for the right to collect guano in Pamitinan.\footcite[113]{Burritt1902} Scientists studied the place. The German geologist Drasche wrote of a journey there. It is interesting that, despite the fact that his work was written entirely in German, he refers to the cave as the \enquote{cueva de S. Mateo.} This would indicate that this had become the official name of the cave. This is the only aspect of Philippine geology which he treats in this fashion; all other geographical features were translated into German.\footcite{Drasche1878}

\subsection{Connecting Pamitinan and Tapusi: Remontado migration}

For tourists what they visited was no more than the cave of San Mateo. For Bonifacio, the mountain and the cave \enquote{of Bernardo Carpio} were named Pamitinan. Julio Nakpil, a commander of troops under Bonifacio and a famed composer, was stationed in the mountains of San Mateo along with Emilio Jacinto. He fought there under the nom-de-guerre of J. Giliw. In his handwritten manuscript, \textit{Apuntes para la historia de la Revoluci\'on Filipina de Teodoro M. Kalaw}, Nakpil wrote of how Bonifacio was fleeing from Aguinaldo in Cavite to San Mateo when he was arrested and executed. Bonifacio's widow, Gregoria de Jesus, was able to escape and reached the San Mateo mountains, joining Nakpil and commanding troops there. She and Nakpil married a year and a half later. Within a month of Bonifacio’s execution, Nakpil composed a dance entitled Pamitinan, which he dedicated to the remontados.\footnote{This manuscript was translated and edited by Encarnacion Alzona as \cite{Alzona1964}. Images of Nakpil's handwritten manuscript and score are included. Relevant material can be found on 12, 45-49, 66 and the score of Pamitinan on 109ff.} This was the tradition which Bonifacio’s Katipunan identified with Pamitinan, the history of resistance to Spanish rule, and not a mythical Tagalog king.

Despite the remarkable differences and great distance between the two places, Pamitinan can be associated with Tapusi both historically and geographically. What were these historical and geographical connections?

Some of the remontados of Tapusi came from San Mateo. Nakpil wrote of the remontados from this region, Rizal did also, refering to \enquote*{los remontados de San Mateo,} in \textit{El Filibusterismo}.\footcite[217]{Rizal1996} The US colonial government in the Philippines conducted a census of the population in 1903. On page 474, in a brief glossary, the census defined \enquote*{nomads} or \enquote*{remontados:}  \enquote{This term refers to a group of wild Tag\'alog people, who tradition says ran away from the town of San Mateo, and whose descendants to-day roam the mountains back of Montalb\'an in association with the Negrito.}\footcite[474]{Census1905}

Linguistic evidence suggests that remontado migration connected Tapusi in the Limutan river valley with Pamitinan. In the 1970s Teodoro Llamzon discovered a new language in Daraitan in the mountainous upstream of Tanay, Rizal. This was exactly where Gironiere located Tapuzi, although he spelled the region \enquote*{Darangitan.} Llamzon designated the language Sinauna (original or ancient), as he considered it represented an ancient strand of Tagalog; the native speakers called their language Tagarug. In the Ethnologue listing of languages it is classified as Remontado Agta. Agta is a language of the Negrito people of the Sierra Madre and the population of Sinauna speakers is supposed to be descended from intermarried remontado and Negrito populations.\footcite[s.v. \enquote{agta, remontado.}]{GordonJr.2005} Sinauna has now been identified as an important transitional form between Tagalog and Bikolano. It is mutually unintelligible with Tagalog.\footnote{\enquote{In fact, we \textins{Llamzon and Rodrigo Dar} did a lexicostatistical analysis of it \textins{Tagarug/sinauna}, Tagalog, and Bicol and found that this was the language that was the missing link in the glottochronological and lexicostatistical numbers from Bisaya to Bicol to Tagalog. In other words, linguists had always noted the consistent degree of difference between Ilonggo and Cebuano and Cebuano and Waray and Waray and Bicol. But the gap from Bicol to Tagalog was so much bigger. Tagarug fit right in between Bicol and Tagalog.} (Rodrigo Dar, 23 Jun 1996, \url{https://groups.google.com/forum/#!topic/soc.culture.filipino/E8nGJSjTPAY} Accessed: May 17, 2009).}

In Southeast Asian linguistics, the pepet vowel is the indifferent vowel; it is akin to schwa. Pepet is the Javanese word for this vowel. Carlos Conant, in his dissertation of 1913, examined the ways in which this vowel differentiated in different languages in the Philippines, e.g. at\textschwa p (roof), becomes atep, atip, atap, and atup.\footcite{Conant1912} Llamzon revisited this thesis and examined the role of dialects in this law. He found that the pepet vowel has not disappeared from most of the languages that Conant claimed had lost the pepet vowel. Conant overlooked the retention of the pepet vowel because he failed to examine dialects.\footcite{Llamzon1976}

For our present purposes, this line in Llamzon's work is important: \enquote{for the Puray dialect, which is geographically located at the back of the Montalban Dam, the regular reflex seems to be \textschwa.} Some brief samples of the dialect follow: ipa, dakip, ngipin, pusod, talong, dikit, dinggin, all of which indicate a retained pepet vowel.\footcite[136]{Llamzon1976} Puray is a river slightly beyond Pamitinan; it is a tributary of the San Mateo River.\footcite[28]{Ugaldezubiaur1880} Thus, in the region of Pamitinan, a Tagalog dialect was spoken which retains the pepet vowel.

That the Limutan river valley was spelled Lumutan in de los Angeles report on Manila's water supply was not an error in transcription; rather, it reflected the ambiguity of the pepet vowel which was retained in both Puray Tagalog and Sinauna. It seems likely that the original semantic significance of the place name was Lumutan (verdant, lush green, mossy). The pepet vowel in the penultimate syllable of a non-enclitic morpheme reflects to i,\footcite{Himes2002} and thus L/*e/mut\textins*{an} came to be L/i/mut\textins*{an} (forgotten).

The Governor of the Province of Rizal wrote on July 8, 1908, in his report to the Governor General of the Philippines,
\begin{quote}
There are several nomad families in the mountains of Tanay called Dagat-dagatan, Lanay, Panusugunan, and others; in the mountains of Antipolo called Uyungan, \textit{Sare}, and \textit{Lumutan}, and others bordering on the barrio of Boso-boso; in the woods and sitios in the jurisdiction of San Mateo and on the Garay River called Pinauran, Cabooan, Lucutan malaqui. These families are estimated to number 1,000 individuals, it being worthy of note that these people come down to the settlements to sell rattan, gugu, wax, bees' honey, and resin in small quantities.\footnote{\cite[415]{ReportWar1908}; Note that \textit{garay} is Sinauna for \enquote*{waterfall.} \parencite[42]{Reid1994}.}
\end{quote}

Lumutan is here adjacent to Sare, which was another name for Tapusi according to de los Angeles. The remontado population according to this report ranged from San Mateo to Lumutan and engaged in trade with the settlements. What Ed. C. De Jesus wrote of the remontados of Cagayan applied to those in Tapusi as well: \enquote{Whatever their original motives for reverting to their old way of life, the remontados quickly found additional reasons for remaining in the mountains and outside of Spanish control. \textit{Contacts among both the Christian towns and the pagan tribes made them the ideal middlemen for the trade between the two groups}.}\footcite[116-7, emphasis added]{DeJesus1980}

The isolationist hypothesis in anthropology has now been discarded; it asserted that hunter gatherer tribal groups had been living without contact with lowland agricultural populations for centuries, even millennia, and had evolved in linguistic and cultural isolation. It is now apparent that Negrito populations in the Philippines established contact and trade with the Austronesians upon the latter's arrival in the Philippines. Trade contact was frequent; the Negritos provided forest products in exchange for agricultural goods. It was trade with Austronesian agriculture which enabled the Negritos to begin settling the jungles and forests of Luzon, which could not provide \enquote{sufficient lipids to supply the nutritional needs of humans in the absence of wild plant starches.}\footcite[47]{Headland1989} The frequent trade was facilitated by the creation of a pidgin, whose core words were derived from the status language, which in this case would have been of Austronesian origin. The pidgin was creolized, and then underwent a lengthy period of de-creolization, as the Negrito creole language adapted to the morphology and syntax of the status language. Thus, the Negritos of the Philippines all speak languages with Austronesian structure and vocabulary. They retain, however, a substrate of non-Austronesian lexemes.\footnote{\cite{Reid1994}; this substrate consists largely of the specialized vocabulary for local biota and ‘secret’ words such as penis, vagina, etc.}

With the arrival of the Spaniards, the forest product for agricultural product trade withered. The upstream Negrito populations were isolated from the downstream rice growers.\footnote{For notions of upstream and downstream communities, see \cite{Bronson1977}.} The remontados, fleeing the Spanish ambit to avoid the onerous impositions of colonialism, became a liminal population which facilitated the resumption of trade between upstream and downstream. In contrast to Gironiere's isolated \enquote*{great phalanstery,} whose female population had never been out of the community, the remontados of Tapusi would have been intensely mobile. They were a population engaged in trade throughout the Southern Sierra Madres, ranging from Tanay to San Mateo. They would have carried on trade with both lowland Tagalogs and with the Umiray Dumaget Negritos. Sinauna would have been the language spoken by the rancher\'ias of the mission of San Andres. When the mission was abandoned in 1700, these Sinauna speakers became known as remontados. To engage in trade it was necessary for them also to speak Tagalog, with which Sinauna is mutually unintelligible. This trade Tagalog of the Sinauna remontados was, it seems likely, the source of the Puray pepet vowel, which is unique among Tagalog dialects and corresponds nicely to the Sinauna language.

The remontados of San Mateo would have passed between the nag-uumpugang bato of Sasocsungan and Pamitinan up the San Mateo River to Tapusi. Linguistic and historical data both establish this connection. Bonifacio and his companions were familiar with the history and legacy of Pamitinan, the history of the remontados. The Carpio legend was the folk memory of this flight. The resistance associated with San Mateo did not consist solely of flight, however.

\subsection{Tulisanes: San Mateo and Banditry}

Colonial authorities labeled the long-standing tradition of resistance at San Mateo banditry, and the inhabitants of the region, \textit{tulisanes}. Telesforo Canseco, the overseer of the Dominican hacienda in Naic, Cavite, wrote of

\begin{quote}
the bandits (\textit{tulisanes}) of San Mateo with long beards whom we have called \textit{tulisan pulpul}, are men who are dedicated to robbing and committing crimes and have taken to the mountains (\textit{remontarse}) and have lived for many years in the mountains of San Mateo, where even the Spanish have not been able to reach them.\footnote{\begin{otherlanguage}{spanish}\enquote{los tulisanes de San Mateo con barbas largas a quien nosotros llam\'abamos Tulisan pulpul o sea hombres que dedicados al robo y a cometer crímenes se han visto precisados a remontarse y vivir muchos anos \textins{\textit{sic}} en los montes de San Mateo, a donde no han podido llegar todav\'ia los espa\~noles \ldots\ }\end{otherlanguage} \parencite[64]{Canseco1999}.\\
Canseco's account was written in 1897 as \enquote{Historia de la insurrecci\'on  filipina en Cavite,} and was housed in the Archivo de la Provincia del Sant\'isimo Rosario de Filipinas, University of Santo Tomas, Manila. The published version is a Spanish-Tagalog diglot. Hernandez notes \enquote{May dalawang uri ng tulisan. Ang una ay tinatawag na \enquote{Dugong Aso,} nagnanakaw at pumapatay. Ang ikalawa naman ay ang \enquote{Tulisang Pulpul,} nagnanakaw subalit tumatakbo at pumapatay lamang o lumalaban kung kailangan.} \cite[69, fn. 7]{Canseco1999} \enquote{There are two classes of tulisan. The first is called \enquote{Dog’s Blood,} they rob and kill. The second is the \enquote{Blunt Tulisan,} they rob but run at kill or fight only if it is necessary.}}
\end{quote}

Noceda and Sanlucar in their 1860 \textit{Vocabulario de la Lengua Tagala} defined tulisan as \enquote{\textit{malhechor, salteador; de tulis, agudo}}/evil-doer, highwayman; from tulis, sharp.\footcite[416, sv. Tulisan]{Noceda1860} The etymon of tulisan is tulis, to sharpen. Tulis is an Austronesian root which developed into the Malay tulisan, writing,\footnote{\cite[62, 224 fn. 8]{Medina2002}; \cite[197, fn. 14]{Borromeo1973}; \cite[1139 sv. tulis]{Quinn2001}. The Proto-Austronesian (PAN) root for sharp is \textit{*Caz\'em}, which reflects to the Tagalog \textit{talim} as well as \textit{tulis}, and to the Malay \textit{tajam}. \textit{Talim} is sharp-edged, while \textit{tulis} is sharp-pointed. This is a much more plausible reconstruction than Laurent Sagart's proposed Proto-Sino-Austronesian (PSAN) root, from which Old Chinese (OC) supposedly derived \textit{*l\textschwa ih}, \enquote{to pencil the eyebrows.}   \cite[96-98]{Ross1995}.} a significance related to the sharpened implement which was used for scratching letters into the leaves of the lontar palm.\footnote{This was not an unusual origin for the word for writing. Both the Latin \textit{scribo} and the Greek \textit{grapho} had an etymological significance of \enquote*{to incise with a sharp point,} while the Sanskrit \textit{likh}, literally meant to scratch.} Tulis thus had pluripotent significance, waiting to be sharpened into one or the other of at least two possible meanings. As the Spaniards supplanted and destroyed Philippine writing systems, the highly literate native populations were driven to orality; tulis came to mean banditry.\footnote{The word tulisan, as banditry, was appropriated by the Spanish. Felix Ramos y Duarte in his 1898 \textit{Diccionario de mejicanismos} defines tulis as \enquote{\textit{ladron, ratero}} (bandit, pickpocket). [\cite[sv. tulis]{RamosyDuarte1898}.]  Tulis, rather than tulisan, had entered Mexican Spanish by the late nineteenth century as a word meaning bandit. The \textit{Diccionario Porrua} attributes the origin of the word \enquote*{tulises} to a \enquote*{grupo de bandoleros del Edo. De Durango} who escaped from the jail of the town of San Andres de Te\'ul, in approximately 1859.  Most notable among them was the famous bandolero, El Cucaracho. [\cite[3013, sv tulises]{Porrua}.] From Te\'ul the dictionary derives the word tulis as bandit. Gironiere, among others, was already using 'tulisan' as a Tagalog word for bandit long before these events in Te\'ul, thus ruling out this etymological reconstruction.\\
An alternative etymology has been proposed by Paloma Albal\'a Hern\'andez in her Americanismos en las Indias del Poniente. She suggests a N\'ahuatl origin for the word, deriving tulis\'an from \enquote{\begin{otherlanguage}{spanish}\textit{tule}, planta de la que se hace el petate, que etimol\'ogicamente procede de la voz n\'ahuatl \textit{tull\'in} o \textit{tolin}, seg\'un Molina (1571) \enquote{juncia o espada\~na} y seg\'un Sim\'eon (1885) \textit{tollin} o \textit{tullin} \enquote{junco, juncia, carrizo}.\end{otherlanguage}} 
\enquote{\textit{tule}, plant from which is made bedrolls, which etymologically proceeds from the n\'ahuatl \textit{tullin} or \textit{tolin}, according to Molina (1571), sedge or bulrush, and according to Sim\'eon (1885) \textit{tollin} or \textit{tullin}, rush, sedge, reed-grass.} [\cite[106, 173]{Hernandez2000}.)]  No further explanation is given for this proposed etymology, but it would seem that petate, bedrolls, were considered a standard item of the bandolero, and since these bedrolls were made from \textit{tule}, the bandoleros became known as tulis. Teresita A. Alcantara, in \cite[6]{Alcantara2008}, follows the same path for the entrance of tulis into Tagalog. This etymology seems far-fetched.\\ 
It would seem likely that the word tulisan traveled from Manila to Acapulco in the final years of the galleon trade. Teul, in the Estado de Durango, was on the west coast of the Mexican isthmus, north of Acapulco. En route, the word also entered Chamorro, as tulisan rather than tulis. Chamorro is an Austronesian language and Chamorro speakers would have found the desinence -an familiar.\\ 
Regardless of the path taken by the word \enquote*{tulisan} in its transpacific peregrination, what is important is that there was a specific historical phenomenon in the nineteenth century in both Mexico and the Philippines with which the word was associated: social banditry.}

Eric Hobsbawm, in his work \textit{Bandits}, writes that social bandits, \enquote{are peasant outlaws whom the lord and state regard as criminals, but who remain within peasant society, and are considered by their people as heroes, as champions, avengers, fighters for justice, perhaps even leaders of liberation, and in any case as men to be admired, helped and supported.}\footcite[20]{Hobsbawm2000} This description aptly matches the phenomenon of tulisanes in the Philippines.

On the subject of tulisanes, Henry Ellis wrote, 

\begin{quote}
Brigandage still exists in Luzon to a considerable extent, armed bands of Tulisanies \textins{\textit{sic}} (hill robbers) patrolling the country levying contributions and plundering with seldom much effectual molestation from the authorities, carrying their depredation in quite an organized form into the suburbs of Manila itself \ldots\ \\
A party of soldiers, under the command of Lieutenant Enciso, in the gray of the morning of the 25th July, managed to surprise a famous brigand leader of the name of Jiminez, who with a part of his band was caught napping in a house \textit{in the neighborhood of the cave of San Mateo} \ldots\ \\
The chief (Jiminez), although in figure an exceedingly slight, small man, had through the daring and determination of his character long held a most perfect sway and control not only over his own particular band but more or less over all the \enquote{gentlemen of the craft} in that part of the country, and, it was said, had frequently used his restraining power for good, punishing severely among his followers acts of wanton outrage and restraining them from unnecessary violence and bloodshed. He carried on a black-mail system, levying contributions principally on the rich, and was not only respected but rather a favourite among the poorer villagers, going amongst them in perfect immunity.\footcite[170-3, emphasis added]{Ellis1859}.
\end{quote}

There is frequently a disjuncture between the reality of banditry and its popular perception. In the late nineteenth century Philippines, banditry was endemic throughout the regions of central and southern Luzon.  Banditry reached its greatest heights in Cavite, where the rise of plantation agriculture brought social conflict into sharp focus. The epicenter of tulisan activity in popular consciousness, however, was the mountains of San Mateo, and Mount Pamitinan in particular.

This tradition of resistance was precisely what the Carpio legend invoked. For \textit{Pasyon and Revolution} to state that the masses live in \enquote{a society where King Bernardo Carpio was no less real than the Spanish governor-general} is to fail completely to understand the function of legend. The geographical specificity of the legend, the insistence upon San Mateo as the location of Bernardo Carpio, served as a metonym for social banditry and resistance to the ruling class.

\subsection{Conflating Pamitinan and Tapusi: elite error}

How then did Tapusi become not merely associated with but actually conflated with Pamitinan and the cave of Bernardo Carpio, if it is a geographically distinct location?

Santiago Alvarez, when speaking of Bonifacio's intention to assault Manila from San Mateo refers to Bonifacio's hiding place in the mountains of San Mateo as \enquote*{Tapusi.}\footcite[156]{Alvarez1992} Alvarez was a mestizo land-owner from Cavite, whose alliance with Bonifacio in opposition to Aguinaldo reflected the continuation of a long-standing regional rivalry between two ruling class factions. His account is an important one for our understanding of the events in Cavite leading up to the arrest and execution of Bonifacio. The greater the remove of an event or person from Alvarez' class and geographical ambit, however, the more tenuous are the facts which Alvarez records on the subject. Thus, when Alvarez writes of Maestrong Sebio, a charismatic leader from Bulacan, he misidentifies him as Eusebio Viola, a wealthy mestizo landowner. Maestrong Sebio was in truth Eusebio Roque, a local school teacher.\footcite[98]{Alvarez1992} Another wealthy Cavite\~no, Carlos Ronquillo, also conflated Tapusi with Pamitinan in his 1898 account of the revolution. There is more involved, however, in Ronquillo's account than simple error.

Fray Mariano Gil, a Spanish priest, revealed the existence of the Katipunan to the colonial authorities after hearing the confession of a wife of one of the members. In his report he stated that the Katipunan was amassing weapons at Tapusi. Tapusi was not a mountain in this report, nor did it have any geographic specificity at all. It was simply a fabled place of resistance. The response of the Spanish authorities was not to rush to San Mateo, but to hunt for Bonifacio and his companions in Tondo. Gil's testimony is not evidence for the conflation of Pamitinan and Tapusi but rather for a hazy fear of Tapusi in the minds of the colonial and religious authorities.

Pedro Paterno, writing his self-aggrandizing memoirs on his role in mediating the pact of Biaknabato, stated, \enquote{I climbed mount Tapusi, with its famous cave, eternal refuge of tulisanes and afterwards lair of General Luciano San Miguel, who afterwards died gloriously at Pugad-Babuy under the fire of American cannons \ldots\ }\footnote{\enquote{\begin{otherlanguage}{spanish}Sub\'i al monte Tapusi, en donde se halla su famosa cueva, eterno refugio de tulisanes y entonces amparadora del General Luciano San Miguel, m\'as tarde muerto gloriosamente en Pugad-Babuy bajo el fuego de los ca\~nones americanos\ldots\ \end{otherlanguage}} \parencite[72]{Paterno1910}.} Thus, in 1910 Paterno identified Tapusi with Pamitinan. It is worth pointing out that Paterno, the extremely wealthy and laughably pretentious Bulake\~no, could not speak passable Tagalog and was carried in a hammock from Manila to Biaknabato and back again. He never went anywhere near Pamitinan and he certainly did not \enquote{climb} anything.\footnote{On Paterno, see Resil B. Mojares' excellent \cite{Mojares2006}; and \cite{Reyes2006}.}

The conflation of Pamitinan and Tapusi occurred among outsiders, those excluded by class from the sociolinguistic register of the peasantry and by spatial and temporal remove from the actual geographical specificity of Pamitinan. Tapusi and Pamitinan were connected, in history and in legend. They were not, however, the same.

On the basis of these conflations, Ileto goes on to identify \enquote*{Mount Tapusi} with Meru, a center of power in Southeast Asian conception.\footcite[39]{Ileto1982} This misses the point entirely. Tapusi was not a mountain, it was not in San Mateo, it had no cave; the idea of Bonifacio's journey being a ritual ascent of Tapusi makes no sense in light of historical evidence.\footnote{This makes even more embarrassing the strange new age academic attempt to \enquote*{revive} this tradition. Consolacion Rustia Alaras, a professor of literature at the University of the Philippines, in her work \cite{Alaras1988}) based on \textit{Pasyon and Revolution}, has advocated the revitalization of the nation through sacred sojourns to \enquote*{Tapusi,} in the steps of Bonifacio, who was a great spiritual leader. She leads these treks every year. These sojourns seem more reminiscent of the wide-eyed jaunts into the wild made by European tourists in the late nineteenth century than anything to do with Bonifacio.}

\subsection{San Mateo: central to Bonifacio's military strategy}

Bonifacio's journey to the cave of San Mateo did place him within a nexus of signification. Bonifacio was aware that this was known as the cave of Bernardo Carpio. He was not awakening a sleeping king, however, nor was he manipulating peasant belief. He was participating in a long-standing history of revolt. There is continuity between social banditry and Bonifacio. This continuity is not to be found in \textit{Pasyon and Revolution}'s atavistic, essentialised counter-rational underside to history, however. It is not a continuity of idiom or ideology. Bonifacio's journey to the cave of San Mateo was an act of identifying with the history of mass resistance of the late nineteenth century.

Based on an awareness of its history, Bonifacio recognized the tactical significance of San Mateo's geography and used it to the advantage of the Katipunan at the beginning of the revolution.

In Pasyon and Revolution we read, \enquote{Bonifacio himself, as Carlos Ronquillo reports, told his followers that their legendary king Bernardo would descend from Mount Tapusi to aid the Katipunan rebels.} (111) The source for this claim is Ronquillo's manuscript, \textit{Ilang Talata tungkol sa Paghihimagsik ng 1896-97}; there is no page number given.

In responding to Milagros Guerrero's critique of his work, Ileto stated

\begin{quote}
In 1897, Carlos Ronquillo, the personal Secretary of Emilio Aguinaldo, in his \enquote{history} of the Katipunan uprising castigated Bonifacio for raising false hopes that an army would descend from Mount Tapusi \enquote{to lead his whole army.} \enquote{This plain falsehood,} writes Ronquillo, \enquote{was a deception or morale booster (pangpalakas lo\'ob) perpetrated by Bonifacio; because at the appointed hour neither men nor arms arrived from Tapusi. Up to now we do not know where this mountain is.}\footcites[217]{Ileto1998a}[12]{Ileto1982a}[27-8]{Ileto1984}
\end{quote}

Ileto used this passage in three separate essays. In each case he cited pages 6 and 21 of Ronquillo's unpublished manuscript. In this paragraph there is no longer any reference to Bernardo Carpio. In his stead is Bonifacio's promise that \enquote{an army} would lead the \enquote{whole army.} Not one of the four citations provides the Tagalog original, aside from the phrase \enquote*{pangpalakas lo\'ob.}

In Ileto's later articles, Ronquillo serves as the example of \enquote{the nationalist \enquote{historian} \ldots\ a believer in enlightened liberalism.} Ronquillo, Ileto states, \enquote{already decried this \enquote{dark underside} of Bonifacio's mentality, adding it to the litany of faults that he felt justified Bonifacio’s execution at the hands of Aguinaldo and the Cavite elite. Hopefully, historians today will not participate in this bloody execution by insisting upon a singular, reductionist reading of the text that comprises our national hero.}\footcite[12]{Ileto1982a} Ileto is not questioning the historical accuracy of Ronquillo's statement; he is rather asking that we consider how the \enquote*{masses} would have perceived Bonifacio's claim that Bernardo Carpio, or an army, would descend from Mount Tapusi. Ronquillo is thus a representative of bad \enquote*{reductionist} historiography; to read history in this fashion is to participate in the murder of Bonifacio.

In 1996, the University of the Philippines press published an excellent edition of Ronquillo's manuscript, thoroughly edited and annotated by Isagani Medina.\footcite{Ronquillo1996} Ileto seems to be paraphrasing a passage and a footnote from the manuscript. Nowhere is there anything that could be considered an exact quotation. The first passage reads

\begin{quote}
Because it had been agreed upon, we stopped and waited for the army of Bonifacio that would be coming from Mount Tapusi and were to be firing and would lead the entire army; however, from the agreed upon time to until daybreak it did not arrive.\footnote{Palibhasa'y salitaan, ay nagagsihinto at inantabayanan ang pulutong ni Bonifacio na manggagaling sa bundok ng Tapusi na pawing barilan na siyang mangunguna sa buong pulutong; subalit nang dumating na ang taning na oras hanggang sa magliliwanag na ang araw ay di dumarating. \parencite[216]{Ronquillo1996}.}
\end{quote}

Ronquillo footnoted Tapusi thus

\begin{quote}
This statement by Bonifacio was a tremendous lie because neither people nor arms were at Tapusi and even he himself did not arrive there. This was just a cruel deception of the people!\footnote{Ang sinasabing ito ni Bonifacio ay isang malaking kasinungalingan pagkat ni tao ni baril ay wala sa Tapusi at ni siya naman ay di nakarating doon. Ito'y isang kalupitang pandaya lamang sa tao! \cite[684, fn. 3]{Ronquillo1996}; the footnote is by Ronquillo, indicated by the initials CVR.}
\end{quote}

The statement \enquote{up to now we do not know where this mountain is} and the parenthetical untranslated phrase, \enquote{pangpalakas lo\'ob,} are both absent from Ronquillo's manuscript. Bernardo Carpio is missing as well. This is not the statement of someone who is detecting the \enquote{dark underside} in Bonifacio's mentality; this is an accusation of poor military leadership and of deception. Bonifacio promised troops and he failed to deliver. This is Ronquillo's accusation. It is also, ironically, a \enquote*{cruel deception.}

Zeus Salazar, in his Agosto 29-30, 1896: \textit{Ang Pagsalakay ni Bonifacio sa Maynila}, examines in detail the claim that \enquote{Bonifacio's plan to attack Manila subsequent to the discovery of the Katipunan was never really carried out.} This planned assault on Manila, \enquote{traditional historians} believed \enquote{was replaced instead with an attack on San Juan del Monte,} a much smaller, less coordinated undertaking.\footcite[96]{Salazar1994} Salazar's examination of the dispatches made by the English, German and French consuls in Manila, in conjunction with the existing historical evidence, convincingly demonstrated that the planned assault did, in fact, occur.

Bonifacio had planned a three-pronged assault on Manila -- from the north, Caloocan, Balintawak and surroundings; from the south, Cavite; and from the east, from the mountains of San Mateo. On the night of August 29, the assault was initiated by Bonifacio's forces from San Mateo, the troops in the north likewise attacked. Cavite did not respond to Bonifacio's orders. Numerous justifications for this failure to follow orders were given in memoirs and subsequent accounts: the orders did not apply to all Katipunan balangays, there was no signal given, the Katipunan lacked the necessary arms.\footnote{For all of these justifications, see \cite[108-11]{Salazar1994}.} No excuse is quite as dramatic -- or as dishonest -- as Ronquillo's bald-faced assertion that \enquote{Bonifacio’s forces never came down from Tapusi.} Ronquillo is certainly attempting to justify the execution of Bonifacio, but not because Bonifacio was part of some irrational, dark \enquote{underside} of Philippine society. When Ronquillo wrote his memoirs, Bonifacio was dead. By calling Bonifacio a liar and a poor leader, Ronquillo was not merely justifying his execution; he was covering over the perfidy of the Cavite elite.

Not only did San Mateo and Pamitinan figure prominently in the initial assault on Manila, they remained a vital center for operations under the leadership of Bonifacio and his fellow Katipuneros. Numerous sources attest to this. 

Mariano Ponce, writing on the 6\textsuperscript{th} of May, 1897, from exile in Hong Kong to Ferdinand Blumentritt, gave notes on details of the revolutionary effort culled from various letters he had received, in particular a letter from a \enquote{rebel camp at Baling-Cupang (San Miguel de Mayumo)} He writes

\begin{quote}
In Pamitinan, in the jurisdiction of Montalban and a half kilometer from it (province of Manila), is one of the best defended Tagalog encampments. Columns proceeded from Manila, Mariquina, Pasig and San Mateo intending to attack it on the 7th and 9th of April; but seeing the situation and defenses of the camp, they retreated a great distance without firing a single shot.\footnote{\begin{otherlanguage}{spanish} En Pamitinan, comprensi\'on de Montalban y a medio kilometro de este (provincia de Manila), hay un campamento tagalo de los mejor defendidos. Columnas procedentes de Manila, Mariquina, Pasig y San Mateo intentaron atacarlo el 7 y el 9 de Abril; pero viendo la situaci\'on y defensa del campamento, se retiraron a gran distancia sin disparar un solo tiro. \end{otherlanguage} \parencite[1-3]{Ponce1932}.}
\end{quote}

Pamitinan was prepared for combat and served as a successful base for the resistance of the Katipunan under Bonifacio's leadership. It continued to serve as a base of armed struggle long after Bonifacio's death.\footnote{It was in San Mateo that US Major General Lawton was killed by troops under the command of General Licerio Geronimo. Geronimo had earlier served in the August 29-30\textsuperscript{th} assault under Bonifacio.}
